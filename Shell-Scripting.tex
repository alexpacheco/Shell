\documentclass[slidestop,mathserif,compress,xcolor=svgnames]{beamer} 
\mode<presentation>
{  
  \setbeamertemplate{background canvas}[vertical shading][bottom=blue!5,top=blue!5]
  \setbeamertemplate{navigation symbols}{}%{\insertsectionnavigationsymbol}
  \usetheme{LSU}
}

\usepackage{pgf,pgfarrows,pgfnodes,pgfautomata,pgfheaps,pgfshade}
\usepackage{amsmath,amssymb,amsfonts,subfigure,pifont}
\usepackage{multirow}
\usepackage{tabularx}
\usepackage{booktabs}
\usepackage{colortbl}
\usepackage{tikz}
\usetikzlibrary{shapes,arrows}
\usetikzlibrary{calc}
\pgfdeclarelayer{background}
\pgfdeclarelayer{foreground}
\pgfsetlayers{background,main,foreground}
\usepackage[latin1]{inputenc}
\usepackage[english]{babel}
\usepackage{hyperref}
\usepackage[normalem]{ulem}
% \usepackage{movie15} 

\hypersetup{
  pdftitle={Shell Scripting}
  pdfauthor={Alexander B. Pacheco, User Services Consultant, Louisiana State University}
}                                                         
\usepackage{times}

\setbeamercovered{dynamic}
\beamersetaveragebackground{DarkBlue!2}
\beamertemplateballitem

\usepackage{times}
\usepackage[T1]{fontenc}
\usepackage{graphicx}



\definecolor{DarkGreen}{rgb}{0.0,0.3,0.0}
\definecolor{Blue}{rgb}{0.0,0.0,0.8} 
\definecolor{dodgerblue}{rgb}{0.1,0.1,1.0}
\definecolor{indigo}{rgb}{0.41,0.1,0.0}
\definecolor{seagreen}{rgb}{0.1,1.0,0.1}
\DeclareSymbolFont{extraup}{U}{zavm}{m}{n}
%\DeclareMathSymbol{\vardiamond}{\mathalpha}{extraup}{87}
\newcommand{\cmark}{\ding{51}}
\newcommand{\xmark}{\ding{55}}
\newcommand{\smark}{\ding{77}}
\newcommand*\vardiamond{\textcolor{tigerspurple}{%
  \ensuremath{\blacklozenge}}}
\newcommand*\up{\textcolor{green}{%
  \ensuremath{\blacktriangle}}}
\newcommand*\down{\textcolor{red}{%
  \ensuremath{\blacktriangledown}}}
\newcommand*\const{\textcolor{darkgray}%
  {\textbf{--}}}
\newcommand*\enter{\tikz[baseline=-0.5ex] \draw[<-] (0,0) -| (0.5,0.1);}


\setbeamercolor{uppercol}{fg=white,bg=red!30!black}%
\setbeamercolor{lowercol}{fg=black,bg=red!15!white}%
\setbeamercolor{uppercol1}{fg=white,bg=blue!30!black}%
\setbeamercolor{lowercol1}{fg=black,bg=blue!15!white}%%
\setbeamercolor{uppercol2}{fg=white,bg=green!30!black}%
\setbeamercolor{lowercol2}{fg=black,bg=green!15!white}%
\newenvironment{colorblock}[4]
{
\setbeamercolor{upperblock}{fg=#1,bg=#2}
\setbeamercolor{lowerblock}{fg=#3,bg=#4}
\begin{beamerboxesrounded}[upper=upperblock,lower=lowerblock,shadow=true]}
{\end{beamerboxesrounded}}
\newenvironment{ablock}[0]
{
\begin{beamerboxesrounded}[upper=uppercol,lower=lowercol,shadow=true]}
{\end{beamerboxesrounded}}
\newenvironment{bblock}[0]
{
\begin{beamerboxesrounded}[upper=uppercol1,lower=lowercol1,shadow=true]}
{\end{beamerboxesrounded}}
\newenvironment{eblock}[0]
{
\begin{beamerboxesrounded}[upper=uppercol2,lower=lowercol2,shadow=true]}
{\end{beamerboxesrounded}}


\title{Shell Scripting}


\author[Alex Pacheco]{\large{Alexander~B.~Pacheco}}
       
%\institute[High Performance Computing @ Louisiana State University - http://www.hpc.lsu.edu] {\inst{}\footnotesize{User Services Consultant\\LSU HPC \& LONI\\sys-help@loni.org}}
\institute[HPC Training: Fall 2012] {\inst{}\footnotesize{User Services Consultant\\LSU HPC \& LONI\\sys-help@loni.org}}

\date[\hfill{February 27, 2013\hspace{2cm}\insertframenumber/\inserttotalframenumber}]{\scriptsize{HPC Training Spring 2013\\Louisiana State University\\Baton Rouge\\February 27, 2013}}
     
\subject{Talks}
\keywords{LONI \& LSU HPC Computing Resources, Bash Shell Scripting}
% This is only inserted into the PDF information catalog. Can be left
% out. 




% If you have a file called "university-logo-filename.xxx", where xxx
% is a graphic format that can be processed by latex or pdflatex,
% resp., then you can add a logo as follows:

% Main Logo on bottom left
\pgfdeclareimage[height=0.55cm]{its-logo}{LONI}
\logo{\pgfuseimage{its-logo}}
% University Logo on top left
\pgfdeclareimage[height=0.55cm]{university-logo}{LSUGeauxPurp}
\tllogo{\pgfuseimage{university-logo}}
% Logo at top right
\pgfdeclareimage[height=0.5cm]{institute-logo}{its-logo}
\trlogo{\pgfuseimage{institute-logo}}
% Logo at bottom right
\pgfdeclareimage[height=0.5cm]{hpc-logo}{cct-logo}
\brlogo{\pgfuseimage{hpc-logo}}

% Delete this, if you do not want the table of contents to pop up at
% the beginning of each subsection:
 \AtBeginSection[]
 {
   \begin{frame}<beamer>
    \frametitle{\small{Outline}}
     \scriptsize
     \tableofcontents[currentsection,currentsubsection]
   \end{frame}
 }

\begin{document}

\frame{\titlepage}

\footnotesize
\begin{frame}[label=toc,squeeze]
  \scriptsize
  \frametitle{\small{Outline}}
  \tableofcontents
\end{frame}

%\part{Introduction}
\section{Overview of Introduction to Linux}
\begin{frame}
  \frametitle{\small Overview: Introduction to Linux}
  \begin{eblock}{What is a SHELL}
    \begin{itemize}
      \item The command line interface is the primary interface to Linux/Unix operating systems.
      \item Shells are how command-line interfaces are implemented in Linux/Unix.
      \item Each shell has varying capabilities and features and the user should choose the shell that best suits their needs.
      \item The shell is simply an application running on top of the kernel and provides a powerful interface to the system.
    \end{itemize}
  \end{eblock}
\end{frame}

\subsection{Types of Shell} 
\begin{frame}
  \frametitle{\small Types of Shell}
  {\scriptsize
  \vspace{-0.25cm}
    \begin{itemize}
      \item[\texttt{sh}]: Bourne Shell
      \begin{enumerate}
        {\scriptsize
          \item[$\vardiamond$] Developed by Stephen Bourne at AT\&T Bell Labs
        }
      \end{enumerate}
      \item[\texttt{csh}]: C Shell
      \begin{enumerate}
        {\scriptsize
          \item[$\vardiamond$] Developed by Bill Joy at University of California, Berkeley
        }
      \end{enumerate}
      \item[\texttt{ksh}]: Korn Shell
      \begin{enumerate}
        {\scriptsize
          \item[$\vardiamond$] Developed by David Korn at AT\&T Bell Labs
          \item[$\vardiamond$] backward-compatible with the Bourne shell and includes many features of the C shell
        }
      \end{enumerate}
      \item[\texttt{bash}]: Bourne Again Shell
      \begin{enumerate}
        {\scriptsize
          \item[$\vardiamond$] Developed by Brian Fox for the GNU Project as a free software replacement for the Bourne shell (sh).
          \item[$\vardiamond$] Default Shell on Linux and Mac OSX
          \item[$\vardiamond$] The name is also descriptive of what it did, bashing together the features of sh, csh and ksh
        }
      \end{enumerate}
      \item[\texttt{tcsh}]: TENEX C Shell
      \begin{enumerate}
        {\scriptsize
          \item[$\vardiamond$] Developed by Ken Greer at Carnegie Mellon University 
          \item[$\vardiamond$] It is essentially the C shell with programmable command line completion, command-line editing, and a few other features.
        }
      \end{enumerate}
    \end{itemize}
  }
\end{frame}

\begin{frame}
  \frametitle{\small Shell Comparison}
  \begin{center}
    \begin{columns}
      \column{0.8\textwidth}
      \begin{tikzpicture}
        \node (tbl) {
          \begin{tabularx}{\textwidth}{cccccc}
            \arrayrulecolor{tigersgold}
            \textcolor{white}{\textbf{Software} }& \textcolor{white}{\textbf{sh}} &\textcolor{white}{\textbf {csh}} & \textcolor{white}{\textbf{tcsh}} & \textcolor{white}{\textbf{ksh}} & \textcolor{white}{\textbf{bash}} \\
            Programming Language\rule{0pt}{3.5ex} & \cmark & \cmark & \cmark & \cmark & \cmark \\
            Shell Variables & \cmark & \cmark & \cmark & \cmark & \cmark \\
            Command alias & \xmark & \cmark & \cmark & \cmark & \cmark \\
            Command history & \xmark & \cmark & \cmark & \cmark & \cmark \\
            Filename completion & \xmark & \smark & \cmark & \smark & \cmark \\
            Command line editing & \xmark & \xmark & \cmark & \smark & \cmark \\
            Job control & \xmark & \cmark & \cmark & \cmark & \cmark \\
            [1.0ex]
        \end{tabularx}};
        \begin{pgfonlayer}{background}
          \draw[rounded corners,top color=blue!30!black,bottom color=blue!10!white,
            draw=tigerspurple!30] ($(tbl.north west)+(0.14,0)$)
          rectangle ($(tbl.north east)-(0.13,0.9)$);
          %\draw[rounded corners,top color=tigersgold,bottom color=tigerspurple!20,
          %   middle color=tigerspurple!20,draw=tigerspurple!20] ($(tbl.south west)
          %   +(0.12,0.5)$) rectangle ($(tbl.south east)-(0.12,0)$);
          \draw[rounded corners,top color=green!5,bottom color=green!5,draw=green!5]
          ($(tbl.north east)-(0.13,0.6)$)
          rectangle ($(tbl.south west)+(0.13,0.2)$);
        \end{pgfonlayer}
      \end{tikzpicture}
      \begin{itemize}
      \item[\cmark]: Yes
      \item[\xmark]: No
      \item[\smark]: Yes, not set by default
      \item[Ref]: {\tiny\url{http://www.cis.rit.edu/class/simg211/unixintro/Shell.html}}
      \end{itemize}
    \end{columns}
  \end{center}
\end{frame}

%\begin{frame}<0>[allowframebreaks]
%  \frametitle{\small Frequently used commands}
%  \begin{itemize}
%    {\scriptsize
%    \item[\texttt{cat}]: Show contents of a file
%    {\tiny \item[$\vardiamond$] \texttt{cat filename}}
%    \item[\texttt{cd}]: Change Directory
%    {\tiny \item[$\vardiamond$] \texttt{cd Tutorials}}
%    \item[\texttt{cp}]: Copy a file
%    {\tiny \item[$\vardiamond$] \texttt{cp file1 file2}}
%    \item[\texttt{mv}]: Move or rename a file
%    {\tiny\item[$\vardiamond$] \texttt{mv file1 file2}}
%    \item[\texttt{ls}]: List files in a directory
%    {\tiny
%    \item[$\vardiamond$] {\tiny\texttt{ls Tutorials}}
%    \item[$\vardiamond$] Options to \texttt{ls}
%    \begin{enumerate}
%      \fontsize{6}{8}\selectfont{
%        \item[-l] show long listing format
%        \item[-a] show hidden files
%        \item[-r] reverse order while sorting
%        \item[-t] show modification times
%      }
%    \end{enumerate}
%    }
%    \item[\texttt{mkdir}]: Create a directory
%    {\tiny\item[$\vardiamond$] \texttt{mkdir dir1}}
%    \framebreak
%    \item[\texttt{rm}]: Remove a file
%    {\tiny
%    \item[$\vardiamond$] \texttt{rm file1 file2}
%    \item[$\vardiamond$] Options to \texttt{rm}
%    \begin{enumerate}
%      \fontsize{6}{8}\selectfont{
%      \item[-i] interactive
%      \item[-r] remove files recursively used to delete directories and its contents
%      \item[-f] force, ignore nonexistent files
%      }
%    \end{enumerate}
%    }
%    \item[\texttt{rmdir}]: Remove a directory
%    {\tiny\item[$\vardiamond$] \texttt{rmdir dir1}}
%    \item[\texttt{file}]: Determine file type
%    \item[\texttt{more}]: Display a file one page at a time
%    \item[\texttt{less}]: Same as \texttt{more} but allow scrolling
%    \item[\texttt{man}]: Access Manual for given application
%    \item[\texttt{vi}]: Edit a file using VI/VIM
%    \item[\texttt{emacs}]: Edit a file using Emacs
%    \item[\texttt{wc}]: Count words, lines and characters in a file
%    \item[\texttt{awk}]: File processing and report generating
%    {\tiny\item[$\vardiamond$] \texttt{awk '\{print \$1\}' file1}}
%    \framebreak
%    \item[\texttt{grep}]: Find lines in a file
%    {\tiny\item[$\vardiamond$] \texttt{grep alias .bashrc}}
%    \item[\texttt{sed}]: Stream Editor
%    {\tiny\item[$\vardiamond$] \texttt{sed 's/home/HOME/g' .bashrc}}
%    \item[\texttt{find}]: Find a file
%    \item[\texttt{ln}]: Link a file to another file
%    {\tiny\item[$\vardiamond$] \texttt{ln -s file1 file2}}
%    \item[\texttt{top}]: Produces an ordered list of running processes
%    \item[\texttt{ps}]: Displays statistics on running processes
%    \item[\texttt{scp}]: secure copy a file/directory between two machines
%    {\tiny\item[$\vardiamond$] \texttt{scp username@host1:/path/to/file1 username@host2:/path/to/file2}}
%    \item[\texttt{sftp}]: connect to another machine using secure ftp
%    \item[\texttt{export}]: export variables to your \texttt{PATH} (\texttt{sh,ksh \& bash} only)
%    {\tiny\item[$\vardiamond$] \texttt{export PATH=/home/apacheco/bin:\$\{PATH\}}}
%    \item[\texttt{setenv}]: equivalent of \texttt{export} for \texttt{csh \& tcsh}
%    {\tiny\item[$\vardiamond$] \texttt{setenv LD\_LIBRARY\_PATH /home/apacheco/lib:\$\{LD\_LIBRARY\_PATH\}}}
%    \framebreak
%    \item[\texttt{alias}]: enables replacement of a word by another string
%    {\tiny\item[$\vardiamond$] \texttt{sh/ksh/bash: alias ll="ls -l"}}
%    {\tiny\item[$\vardiamond$] \texttt{csh/tcsh: alias rm "rm -i"}}
%    \item[\texttt{set}]: manipulate environment variables
%    {\tiny\item[$\vardiamond$] \texttt{set -o emacs}}
%    \item[\texttt{echo}]: print to screen or standard output
%    {\tiny\item[$\vardiamond$] \texttt{echo \$LD\_INCLUDE\_PATH}}
%    \item[\texttt{date}]: display or set date and time
%    \item[\texttt{\&}]: run a job in background
%    \item[\texttt{CNTRL-Z}]: suspend a running job
%    \item[{\texttt{CNTRL-C}}]: Kill a running job
%    \item[{\texttt{jobs}}]: Show list of background jobs
%    \item[{\texttt{fg}}]: run a suspended job in foreground
%    \item[{\texttt{bg}}]: run a suspended job in background
%    \item[{\texttt{wait}}]: wait until all backgrounded jobs have completed
%    \item[\texttt{kill}]: kill a running job, need to provide process id
%    }
%  \end{itemize}
%  To learn more about these commands, type \texttt{man \textit{command}} on the command prompt
%\end{frame}

\subsection{File Editing}
\begin{frame}
  \frametitle{\small File Editing}
  \begin{itemize}
    \item The two most commonly used editors on Linux/Unix systems are:
    \begin{enumerate}
      \item \texttt{vi}
      \item \texttt{emacs}
    \end{enumerate}
    \item \texttt{vi} is installed by default on Linux/Unix systems and has only a command line interface (CLI).
    \item \texttt{emacs} has both a CLI and a graphical user interface (GUI).
    \item[$\vardiamond$] If \texttt{emacs} GUI is installed then use \texttt{emacs -nw} to open file in console.
    \item Other editors that you may come across on *nix systems
    \begin{enumerate}
      \item \texttt{kate}: {\scriptsize default editor for KDE.}
      \item \texttt{gedit}: {\scriptsize default text editor for GNOME desktop environment.}
      \item \texttt{gvim}: {\scriptsize GUI version of }\texttt{vim}
      \item \texttt{pico}: {\scriptsize console based plain text editor }
      \item \texttt{nano}: {\scriptsize GNU.org clone of }\texttt{pico}
      \item \texttt{kwrite}: {\scriptsize editor by KDE.}
    \end{enumerate}
    \item {\color{red}You are required to know how to create and edit files for this tutorial.}
  \end{itemize}
\end{frame}

\begin{frame}[allowframebreaks]
  \frametitle{\small Editor Cheatsheets}
% Page 1
  \vspace{-0.6cm}
  {\scriptsize
  \begin{columns}
    \column{0.4\textwidth}
    \begin{eblock}{Cursor Movement}
    \begin{itemize}
      \item move left 
      \item move down
      \item move up
      \item move right
      \item jump to beginning of line
      \item jump to end of line
      \item goto line \texttt{n}
      \item goto top of file
      \item goto end of file
      \item move one page up
      \item move one page down
    \end{itemize}
    \end{eblock}
    \column{0.2\textwidth}
    \begin{eblock}{vi}
    \begin{itemize}
      \item \texttt{h}
      \item \texttt{j}
      \item \texttt{k}
      \item \texttt{l}
      \item \texttt{\^}
      \item \texttt{\$}
      \item \texttt{nG}
      \item \texttt{1G}
      \item \texttt{G}
      \item \texttt{C-u}
      \item \texttt{C-d}
    \end{itemize}
    \end{eblock}
    \column{0.4\textwidth}
    \begin{eblock}{emacs}
    \begin{itemize}
      \item \texttt{C-b}
      \item \texttt{C-n}
      \item \texttt{C-p}
      \item \texttt{C-f}
      \item \texttt{C-a}
      \item \texttt{C-e}
      \item \texttt{M-x goto-line [RET] n}
      \item \texttt{M-<}
      \item \texttt{M->}
      \item \texttt{M-v}
      \item \texttt{C-v}
    \end{itemize}
    \end{eblock}
  \end{columns}
  }
  \vspace{-0.1cm}
% Page 2
  \begin{columns}
    \column{0.6\textwidth}
    \begin{itemize}
      {\scriptsize
      \item[C]: Control Key
      \item[M]: Meta or ESCAPE (ESC) Key
      \item[{[RET]}]: Enter Key
      }
    \end{itemize}
  \end{columns}
  \framebreak
  {\scriptsize
  \begin{columns}
    \column{0.6\textwidth}
    \vspace{-0.5cm}
    \begin{eblock}{Insert/Appending Text}
    \begin{itemize}
      \item insert at cursor 
      \item insert at beginning of line
      \item append after cursor
      \item append at end of line
      \item newline after cursor in insert mode
      \item newline before cursor in insert mode
      \item append at end of line
      \item exit insert mode
    \end{itemize}
    \end{eblock}
    \column{0.2\textwidth}
    \vspace{-0.5cm}
    \begin{eblock}{vi}
    \begin{itemize}
      \item \texttt{i}
      \item \texttt{I}
      \item \texttt{a}
      \item \texttt{A}
      \item \texttt{o}
      \item \texttt{O}
      \item \texttt{ea}
      \item \texttt{ESC}
    \end{itemize}
    \end{eblock}
  \end{columns}
  }
  \begin{itemize}
    \item {\scriptsize \texttt{emacs} has only one mode unlike \texttt{vi} which has insert and command mode}
  \end{itemize}
  \framebreak
  {\scriptsize
   \begin{columns}
    \column{0.5\textwidth}
     \vspace{-0.5cm}
    \begin{eblock}{File Editing}
    \begin{itemize}
      \item save file
      \item save file and exit
      \item quit
      \item quit without saving
      \item delete a line
      \item delete \texttt{\textit{n}} lines
      \item paste deleted line after cursor
      \item paste before cursor
      \item undo edit
      \item delete from cursor to end of line
      \item search forward for \textit{patt}
      \item search backward for \textit{patt}
      \item search again forward (backward)
    \end{itemize}
    \end{eblock}
    \column{0.25\textwidth}
     \vspace{-0.5cm}
    \begin{eblock}{vi}
    \begin{itemize}
      \item \texttt{:w}
      \item \texttt{:wq, ZZ}
      \item \texttt{:q}
      \item \texttt{:q!}
      \item \texttt{dd}
      \item \texttt{\textit{n}dd}
      \item \texttt{p}
      \item \texttt{P}
      \item \texttt{u}
      \item \texttt{D}
      \item \texttt{$\backslash$\textit{patt}}
      \item \texttt{?\textit{patt}}
      \item \texttt{n}
    \end{itemize}
    \end{eblock}
    \column{0.3\textwidth}
     \vspace{-0.5cm}
    \begin{eblock}{emacs}
    \begin{itemize}
      \item \texttt{C-x C-s}
      \item \texttt{}
      \item \texttt{C-x C-c}
      \item \texttt{}
      \item \texttt{C-a C-k}
      \item \texttt{C-a M-\textit{n} C-k}
      \item \texttt{C-y}
      \item
      \item \texttt{C-\_}
      \item \texttt{C-k}
      \item \texttt{C-s \textit{patt}}
      \item \texttt{C-r \textit{patt}}
      \item \texttt{C-s(r)}
    \end{itemize}
    \end{eblock}
  \end{columns}
  }
  \framebreak
  {\scriptsize
  \begin{columns}
    \column{0.4\textwidth}
     \vspace{-0.5cm}
    \begin{eblock}{File Editing (contd)}
    \begin{itemize}
      \item replace a character
      \item join next line to current
      \item change a line
      \item change a word
      \item change to end of line
      \item delete a character
      \item delete a word
      \item edit/open file \textit{file}
      \item insert file \textit{file}
      \item split window horizontally
      \item split window vertically
      \item switch windows
    \end{itemize}
    \end{eblock}
    \column{0.35\textwidth}
     \vspace{-0.5cm}
    \begin{eblock}{vi}
    \begin{itemize}
      \item \texttt{r}
      \item \texttt{J}
      \item \texttt{cc}
      \item \texttt{cw}
      \item \texttt{c\$}
      \item \texttt{x}
      \item \texttt{dw}
      \item \texttt{:e \texttt{file}}
      \item \texttt{:r \texttt{file}}
      \item \texttt{:split or C-ws}
      \item \texttt{:vsplit or C-wv}
      \item \texttt{C-ww}
    \end{itemize}
    \end{eblock}
    \column{0.3\textwidth}
     \vspace{-0.5cm}
    \begin{eblock}{emacs}
    \begin{itemize}
      \item 
      \item 
      \item 
      \item
      \item
      \item \texttt{C-d}
      \item \texttt{M-d}
      \item \texttt{C-x C-f \textit{file}}
      \item \texttt{C-x i \textit{file}}
      \item \texttt{C-x 2}
      \item \texttt{C-x 3}
      \item \texttt{C-x o}
    \end{itemize}
    \end{eblock}
  \end{columns}
  }
  \begin{itemize}
    \item {\tiny To change a line or word in \texttt{emacs}, use \texttt{C-\textit{spacebar}} and navigate to end of word or line to select text and then delete using \texttt{C-w}}
  \end{itemize}
  \framebreak
  \begin{itemize}
    \item Do a google search for more detailed cheatsheets
    \item[\texttt{vi}] \url{https://www.google.com/search?q=vi+cheatsheet}
    \item[\texttt{emacs}] \url{https://www.google.com/search?q=emacs+cheatsheet}
  \end{itemize}
\end{frame}

\subsection{Variables}
\begin{frame}[fragile,allowframebreaks]
  \frametitle{\small Variables}
  \begin{itemize}
    \item *nix also permits the use of variables, similar to any programming language such as C, C++, Fortran etc
    \item A variable is a named object that contains data used by one or more applications. 
    \item There are two types of variables, Environment and User Defined and can contain  a number, character or a string of characters.
    \item Environment Variables provides a simple way to share configuration settings between multiple applications and processes in Linux.
    \item By Convention, enviromental variables are often named using all uppercase letters
    \item[e.g.] \texttt{PATH, LD\_LIBRARY\_PATH, LD\_INCLUDE\_PATH, TEXINPUTS, etc}
    \item To reference a variable (environment or user defined) prepend \$ to the name of the variable
    \item[e.g.] \$PATH, \$LD\_LIBRARY\_PATH
%    \framebreak
%    \item The command \textbf{printenv} list the current environmental variables.
%    \item[$\bigstar$] \texttt{Type \textbf{printenv} on your command prompt to list all environment variables in your current session.}
%    \item The command \textbf{env} is used to either print a list of environment variables or run another utility in an altered environment without having to modify the currently existing environment.
%    \item[$\bigstar$] \texttt{Type \textbf{env SHELL=/bin/tcsh xterm} to start an xterm session in tcsh}
%    \item[$\vardiamond$] \texttt{To execute the above command successfully, you need to be in GUI mode on the virtual OS or logged into a remote systems with X-Forwarding enabled.}
  \end{itemize}
%  \framebreak
%  \begin{description}
%    \item[PATH:] A list of directory paths.
%    \item[HOME:] indicate where a user's home directory is located in the file system.
%    \item[PWD:] contains path to current working directory.
%    \item[OLDPWD:] contains path to previous working directory.
%    \item[TERM:] specifies the type of computer terminal or terminal emulator being used
%    \item[SHELL:] contains name of the running, interactive shell.
%    \item[PS1:] default command prompt
%    \item[PS2:] secondary command prompt
%    \item[LD\_LIBRARY\_PATH:] colon-separated set of directories where libraries should be searched for first
%    \item[HOSTNAME:] The systems host name
%    \item[USER:] Current logged in user's name
%    \item[DISPLAY:] Network name of the X11 display to connect to, if available.
%    \item[LD\_INCLUDE\_PATH:] 
%  \end{description}
  \framebreak
  \begin{itemize}
    \item You can edit the environment variables.
    \item Command to do this depends on the shell
    \item[$\bigstar$] \texttt{To add your bin directory to the PATH variable}
    \item[] \texttt{sh/ksh/bash}: \textbf{export PATH=\$\{HOME\}/bin:\$\{PATH\}}
    \item[] \texttt{csh/tcsh}: \textbf{setenv PATH\,\,\$\{HOME\}/bin:\$\{PATH\}}
    \item[$\bigstar$] \texttt{Note the syntax for the above commands}
    \item[$\bigstar$] \textbf{sh/ksh/bash:} \texttt{no spaces except between export and PATH}
    \item[$\bigstar$] \textbf{csh,tcsh:} \texttt{no = sign, just a space between PATH and the absolute path}
    \item[$\bigstar$] \textbf{all shells:} \texttt{colon(:) to separate different paths and the variable that is appended to}
    \item \textbf{Yes, the order matters.} If you have a customized version of a software say perl in your home directory, if you append the perl path to \$PATH at the end, your program will use the system wide perl not your locally installed version.
    \item Rules for Variable Names
    \begin{enumerate}
      \fontsize{7}{10}\selectfont{
        \item Variable names must start with a letter or underscore
        \item Number can be used anywhere else
        \item DO NOT USE special characters such as \texttt{@, \#, \%, \$}
        \item Case sensitive
        \item Examples
        \begin{itemize}
          \fontsize{7}{10}\selectfont{
          \item Allowed: VARIABLE, VAR1234able, var\_name, \_VAR
          \item Not Allowed: 1VARIABLE, \%NAME, \$myvar, VAR@NAME 
          }
        \end{itemize}
      }
    \end{enumerate}
    \item Assigning value to a variable
    \begin{center}
      \begin{tikzpicture}
        \node (tbl) {
          \begin{tabularx}{0.8\textwidth}{ccc}
            \arrayrulecolor{tigersgold}
            \textcolor{white}{\textbf{Type} }& \textcolor{white}{\textbf{sh,ksh,bash}} &\textcolor{white}{\textbf {csh,tcsh}}\\
            Shell \rule{0pt}{3.5ex} & name=value & set name = value \\
            Environment & export name=value & setenv name value \\
            [1.0ex]
        \end{tabularx}};
        \begin{pgfonlayer}{background}
          \draw[rounded corners,top color=blue!30!black,bottom color=blue!10!white,
            draw=tigerspurple!30] ($(tbl.north west)+(0.14,0)$)
          rectangle ($(tbl.north east)-(0.13,0.9)$);
          \draw[rounded corners,top color=green!5,bottom color=green!5,draw=green!5]
          ($(tbl.north east)-(0.13,0.6)$)
          rectangle ($(tbl.south west)+(0.13,0.2)$);
        \end{pgfonlayer}
      \end{tikzpicture}
    \end{center}
    \item \textbf{sh,ksh,bash} THERE IS NO SPACE ON EITHER SIDE OF =
    \item \textbf{csh,tcsh} space on either side of = is allowed for the \texttt{set} command
    \item \textbf{csh,tcsh} There is no = in the \texttt{setenv} command
  \end{itemize}
\end{frame}

\subsection{File Permissions}
\begin{frame}[fragile, allowframebreaks]
  \frametitle{\small File Permissions}
  \begin{itemize}
    \item In *NIX OS's, you have three types of file permissions
    \begin{enumerate}
      {\footnotesize
        \item read (r)
        \item write (w)
        \item execute (x)
      }
    \end{enumerate}
    \item for three types of users
    \begin{enumerate}
      {\footnotesize
        \item user
        \item group
        \item world i.e. everyone else who has access to the system
      }
    \end{enumerate}
    \begin{eblock}{}
      \begin{tabular}{lllllllll}
        drwxr-xr-x. & 2 & user & user & 4096 & Jan & 28 & 08:27 & Public\\
        -rw-rw-r-\,-. & 1 & user & user & 3047 & Jan & 28 & 09:34 & README\\
      \end{tabular}
    \end{eblock}
    \item The first character signifies the type of the file
    \item[] \texttt{d} for directory
    \item[] \texttt{l} for symbolic link
    \item[] \texttt{-} for normal file
    \item The next three characters of first triad signifies what the owner can do
    \item The second triad signifies what group member can do
    \item The third triad signifies what everyone else can do
    \item Read carries a weight of 4
    \item Write carries a weight of 2
    \item Execute carries a weight of 1
    \item The weights are added to give a value of 7 (rwx), 6(rw), 5(rx) or 3(wx) permissions. 
    \item \textbf{chmod} is a *NIX command to change permissions on a file
    \item To give user rwx, group rx and world x permission, the command is
    \item[] \texttt{chmod 751 filename}
    \framebreak
    \item Instead of using numerical permissions you can also use symbolic mode
    \item[u/g/o or a] user/group/world or all i.e. ugo
    \item[+/-] Add/remove permission
    \item[r/w/x] read/write/execute
    \item Give everyone execute permission: 
    \item[] \texttt{chmod a+x hello.sh }
    \item[] \texttt{chmod ugo+x hello.sh}
    \item Remove group and world read \& write permission: 
    \item[] \texttt{chmod go-rw hello.sh}
    \item Use the \texttt{-R} flag to change permissions recursively, all files and directories and their contents.
    \item[] \texttt{chmod -R 755 \$\{HOME\}/*}
    \item[] What is the permission on \$\{HOME\}?
  \end{itemize}
\end{frame}

\subsection{Input and Output}
\begin{frame}[fragile,allowframebreaks]
  \frametitle{\small Input/Output}
  \begin{itemize}
    \item The command \textbf{echo} is used for displaying output to screen 
    \item For reading input from screen/keyboard/prompt
    \item[\textbf{bash}] \textbf{read}
    \item[\textbf{tcsh}] \textbf{\$<}
    \item The \textbf{read} statement takes all characters typed until the \enter \,key is pressed and stores them into a variable.
    \item[Syntax] \texttt{read <variable name>}
    \item[Example] \texttt{read name\enter}
    \item[] \textit{Alex Pacheco}
    \item \textbf{\$<} can accept only one argument. If you have multiple arguments, enclose the \textbf{\$<} within quotes e.g. \textbf{"\$<"}
    \item[Syntax:] \texttt{set <variable> = \$<}
    \item[Example:] \texttt{set name = "\$<"\enter}
    \item[] \textit{Alex Pacheco}
    \item In the above examples, the name that you enter in stored in the variable \texttt{name}.
    \item Use the \textbf{echo} command to print the variable \texttt{name} to the screen
    \item[] \texttt{echo \$name\enter}
    \item The \textbf{echo} statement can print multiple arguments. 
    \item By default, \textbf{echo} eliminates redundant whitespace (multiple spaces and tabs) and replaces it with a single whitespace between arguments. 
    \item To include redundant whitespace, enclose the arguments within double quotes
    \item[Example:] \texttt{echo Welcome to HPC\quad\quad Training}\enter\, (more than one space between HPC and Training
    \item[] \texttt{echo "Welcome to HPC\quad\quad Training"}\enter
    \item[] \texttt{read name\enter} or \texttt{set name = "\$<"\enter}
    \item[] \textit{Alex\quad\quad\quad Pacheco\enter}
    \item[] \texttt{echo \$name\enter}
    \item[] \texttt{echo "\$name"\enter}
    \framebreak
    \item You can also use the \textbf{printf} command to display output
    \item[Usage:] \texttt{printf <format> <arguments>}
    \item[Examples:] \texttt{printf "\$name"\enter}
    \item[] \texttt{printf "\%s$\backslash$n" "\$name"\enter}
    \item Format Descriptors
    \begin{enumerate}
      {\scriptsize
        \item[\%s] print argument as a string
        \item[\%d] print argument as an integer
        \item[\%f] print argument as a floating point number
        \item[$\backslash$n] print new line
        \item[] you can add a width for the argument between the \% and \{s,d,f\} fields
        \item[] \%4s, \%5d, \%7.4f
      }
    \end{enumerate}
    \item The \textbf{printf} command is used in \textbf{awk} to print formatted data (more on this later) 
  \end{itemize}
\end{frame}

\begin{frame}
  \frametitle{\small I/O Redirection}
  \begin{itemize}
    \item There are three file descriptors for I/O streams
    \begin{enumerate}
      {\footnotesize
        \item STDIN: Standard Input
        \item STDOUT: Standard Output
        \item STDERR: Standard Error
      }
    \end{enumerate}
    \item 1 represents STDOUT and 2 represents STDOUT
    \item I/O redirection allows users to connect applications
    \begin{itemize}
      {\footnotesize
      \item[<]: connects a file to STDIN of an application
      \item[>]: connects STDOUT of an application to a file
      \item[> >]: connects STDOUT of an application by appending to a file
      \item[|]: connects the STDOUT of an application to STDIN of another application.
      }
    \end{itemize}
    \item Examples:
    \begin{enumerate}
      {\footnotesize
      \item write STDOUT to file: \texttt{ls -l > ls-l.out }
      \item write STDERR to file: \texttt{ls -l 2> ls-l.err }
      \item write STDOUT to STDERR: \texttt{ls -l 1>\&2 }
      \item write STDERR to STDOUT: \texttt{ls -l 2>\&1 }
      \item send STDOUT as STDIN: \texttt{ls -l | wc -l}
      }
    \end{enumerate}
  \end{itemize}
\end{frame}

%\begin{frame}{\small Exercise}
%  \begin{eblock}{}
%    \begin{itemize}
%      \item Create two shell variables containing
%      \begin{enumerate}
%        {\footnotesize
%          \item your name
%          \item[e.g.] MYNAME=Alex
%          \item a standard greeting
%          \item[e.g.] Greet=Hello
%          \item[] \texttt{echo \$MYNAME}\enter
%          \item[] \texttt{echo \$Greet}\enter
%          \item Combine and print your name and the greeting
%          \item[] \texttt{echo \$Greet \$MYNAME}\enter
%          \item What is the output of the following command?
%          \item[] \texttt{echo \$Greet \$MYNAME, Welcome to HPC Training}\enter
%        }
%      \end{enumerate}
%    \end{itemize}
%  \end{eblock}
%\end{frame}

\section{Shell Scripting Basics}
\subsection{Start Up Scripts}
\begin{frame}
  \frametitle{\small Start Up Scripts}
  \begin{itemize}
    {\scriptsize
    \item When you login to a *NIX computer, shell scripts are automatically loaded depending on your default \texttt{shell}
    \item \textbf{sh,ksh}
    \begin{enumerate}
      {\scriptsize
        \item \texttt{/etc/profile}
        \item \texttt{\$HOME/.profile}
      }
    \end{enumerate}
    \item \textbf{bash}
    \begin{enumerate}
      {\scriptsize
        \item \texttt{/etc/profile, login terminal only}
        \item \texttt{/etc/bashrc or /etc/bash/bashrc}
        \item \texttt{\$HOME/.bash\_profile, login terminal only}
        \item \texttt{\$HOME/.bashrc}
      }
    \end{enumerate}
    \item \textbf{csh,tcsh}
    \begin{enumerate}
      {\scriptsize
        \item \texttt{/etc/csh.cshrc}
        \item \texttt{\$HOME/.tcshrc}
        \item \texttt{\$HOME/.cshrc if .tcshrc is not present}
      }
    \end{enumerate}
    \item The \texttt{.bashrc, .tcshrc, .cshrc, .bash\_profile} are script files where users can define their own aliases, environment variables, modify paths etc.
    \item e.g. the \texttt{alias rm="rm -i"} command will modify all \texttt{rm} commands that you type as \texttt{rm -i} 
    }
  \end{itemize}
\end{frame}
\begin{frame}[fragile, allowframebreaks]
  \frametitle{\small Examples}
  {\tiny
    \begin{bblock}{.bashrc}
      \begin{verbatim}
# .bashrc

# Source global definitions
if [ -f /etc/bashrc ]; then
        . /etc/bashrc
fi

# User specific aliases and functions
alias c="clear"
alias rm="/bin/rm -i"
alias psu="ps -u apacheco"
alias em="emacs -nw"
alias ll="ls -lF"
alias la="ls -al"
export PATH=/home/apacheco/bin:${PATH}
export g09root=/home/apacheco/Software/Gaussian09
export GAUSS_SCRDIR=/home/apacheco/Software/scratch
source $g09root/g09/bsd/g09.profile

export TEXINPUTS=.:/usr/share/texmf//:/home/apacheco/LaTeX//:${TEXINPUTS}
export BIBINPUTS=.:/home/apacheco/TeX//:${BIBINPUTS}
      \end{verbatim}
    \end{bblock}
    \begin{bblock}{.tcshrc}
      \begin{verbatim}
# .tcshrc

# User specific aliases and functions
alias c clear
alias rm "/bin/rm -i"
alias psu "ps -u apacheco"
alias em "emacs -nw"
alias ll "ls -lF"
alias la "ls -al"
setenv PATH "/home/apacheco/bin:${PATH}"
setenv g09root "/home/apacheco/Software/Gaussian09"
setenv GAUSS_SCRDIR "/home/apacheco/Software/scratch"
source $g09root/g09/bsd/g09.login

setenv TEXINPUTS ".:/usr/share/texmf//:/home/apacheco/LaTeX//:${TEXINPUTS}"
setenv BIBINPUTS ".:/home/apacheco/TeX//:${BIBINPUTS}"
      \end{verbatim}
    \end{bblock}
  }
\end{frame}

\begin{frame}
  \frametitle{\small What is a scripting Language?}
%  \vspace{-0.25cm}
  \begin{eblock}{}%{What is a Scripting Language?}
    {\scriptsize
    \begin{itemize}
      \item A \textbf{scripting language} or \textbf{script language} is a \emph{programming language} that supports the writing of \textbf{scripts}.
      \item \textbf{Scripting Languages} provide a higher level of abstraction than standard programming languages.
      \item Compared to programming languages, scripting languages do not distinguish between data types: integers, real values, strings, etc.
      \item Scripting Languages tend to be good for automating the execution of other programs.
      \begin{enumerate}
        {\scriptsize
          \item[$\vardiamond$] analyzing data
          \item[$\vardiamond$] running daily backups
        }
      \end{enumerate}
      \item They are also good for writing a program that is going to be used only once and then discarded.
%    \end{itemize}
%  \end{eblock}
%  \begin{eblock}{What is a script?}
%    \begin{itemize}
      \item A \textbf{script} is a program written for a software environment that automate the execution of tasks which could alternatively be executed one-by-one by a human operator.
      \item The majority of script programs are ``quick and dirty'', where the main goal is to get the program written quickly.
    \end{itemize}
    }
  \end{eblock}
\end{frame}

\subsection{Getting Started with Writing Simple Scripts}
\begin{frame}[fragile]
  \frametitle{\small Writing your first script}
  \begin{bblock}{Three things to do to write and execute a script}
    {\scriptsize
    \begin{enumerate}
      \item Write a script
      \begin{itemize}
        {\scriptsize
        \item A shell script is a file that contains ASCII text. 
        \item Create a file, \texttt{hello.sh} with the following lines 
        }
      \end{itemize}
      {\tiny
      \begin{verbatim}
#!/bin/bash
# My First Script
echo "Hello World!"
      \end{verbatim}
      }
      \item Set permissions
      {\tiny
      \begin{verbatim}
~/Tutorials/BASH/scripts> chmod 755 hello.sh 
      \end{verbatim}
      }
      \item Execute the script
      {\tiny
      \begin{verbatim}
~/Tutorials/BASH/scripts> ./hello.sh 
Hello World!
      \end{verbatim}
      }
    \end{enumerate}
    }
  \end{bblock}
\end{frame}

\begin{frame}[fragile]
  \frametitle{\small Description of the script}
  \fontsize{7}{9}\selectfont{
  \begin{itemize}
    \item My First Script
  \begin{verbatim}
#!/bin/bash
# My First Script
echo "Hello World!"
  \end{verbatim}
    \item The first line is called the "SheBang'' line. It tells the OS which interpreter to use. In the current example, bash
    \item Other options are:
    \begin{enumerate}
      {\scriptsize
        \item[$\vardiamond$] \texttt{sh\quad\,:   \#!/bin/sh}
        \item[$\vardiamond$] \texttt{ksh :  \#!/bin/ksh}
        \item[$\vardiamond$] \texttt{csh :  \#!/bin/csh}
        \item[$\vardiamond$] \texttt{tcsh: \#!/bin/tcsh}
      }
    \end{enumerate}
    \item The second line is a comment. All comments begin with "\#".
    \item The third line tells the OS to print "Hello World!" to the screen.
  \end{itemize}
  }
\end{frame}

\begin{frame}[allowframebreaks]
  \frametitle{\small Special Characters}
  \fontsize{7}{9}\selectfont{
  \begin{description}
    \item[\#:] starts a comment.
    \item[\$:] indicates the name of a variable.
    \item[$\backslash$:] escape character to display next character literally.
    \item[\{ \}:] used to enclose name of variable.
    \item[;] Command separator [semicolon]. Permits putting two or more commands on the same line.
    \item[;;] Terminator in a case option [double semicolon].
    \item[.] "dot" command [period]. Equivalent to source. This is a bash builtin.
    \item[\$?] exit status variable.
    \item[\$\$] process ID variable.
    \item[{[ ]}] test expression
    \item[{[[ ]]}] test expression, more flexible than [ ] 
    \item[{\$[ ], (( ))}] integer expansion
    \item[||, \&\&, !] Logical OR, AND and NOT
  \end{description}
  }
\end{frame}

\begin{frame}[fragile,allowframebreaks]
  \frametitle{\small Quotation}
  \begin{itemize}
    \item Double Quotation \texttt{" "}
    \begin{itemize}
      {\scriptsize
        \item Enclosed string is expanded ("\$", "/" and "`")
        \item Example: \texttt{echo "\$myvar"} prints the value of \texttt{myvar}
      }
    \end{itemize}
    \item Single Quotation \texttt{' '}
    \begin{itemize}
      {\scriptsize
        \item Enclosed string is read literally
        \item Example: \texttt{echo '\$myvar'} prints \texttt{\$myvar}
      }
    \end{itemize}
    \item Back Quotation \texttt{` `}
    \begin{itemize}
      {\scriptsize
        \item Used for command substitution
        \item Enclosed string is executed as a command
        \item Example: \texttt{echo `pwd`} prints the output of the \texttt{pwd} command i.e. print working directory
        \item In \textbf{bash}, you can also use \texttt{\$($\cdots$)} instead of \texttt{`$\cdots$`}
        \item[] e.g. \texttt{\$(pwd)} and \texttt{`pwd`} are the same
      }
    \end{itemize}
  \end{itemize}
%  \fontsize{9}{8}{
  \tiny{
  \begin{verbatim}
~/Tutorials/BASH/scripts> cat quotes.sh 
#!/bin/bash

HI=Hello

echo HI             # displays HI
echo $HI            # displays Hello
echo \$HI           # displays $HI
echo "$HI"          # displays Hello
echo '$HI'          # displays $HI
echo "$HIAlex"      # displays nothing
echo "${HI}Alex"    # displays HelloAlex
echo `pwd`          # displays working directory
echo $(pwd)         # displays working directory
~/Tutorials/BASH/scripts> ./quotes.sh 
HI
Hello
$HI
Hello
$HI

HelloAlex
/home/apacheco/Tutorials/BASH/scripts
/home/apacheco/Tutorials/BASH/scripts
~/Tutorials/BASH/scripts>
  \end{verbatim}
  }
\end{frame}

\begin{frame}{\small Exercises}
  \begin{itemize}
    \item Create shell scripts to do the following
    \begin{enumerate}
      {\footnotesize
        \item Write a simple hello world script
        \item Modify the above script to use a variable
        \item Modify the above script to prompt you for your name and then display your name with a greeting.
      }
    \end{enumerate}
  \end{itemize}
\end{frame}

\begin{frame}[fragile]{\small Solution}
  \begin{columns}
    \column{0.5\textwidth}
    \begin{eblock}{}
      {\fontsize{4.5}{5.5}\selectfont
        \begin{verbatim}
~/Tutorials/BASH/scripts> cat hellovariable.sh
#!/bin/bash

# Hello World script using a variable
STR="Hello World!"
echo $STR
        \end{verbatim}
      }
    \end{eblock}
    \column{0.5\textwidth}
    \begin{eblock}{}
      {\fontsize{4.5}{5.5}\selectfont
        \begin{verbatim}
~/Tutorials/BASH/scripts> ./hellovariable.sh 
Hello World!
        \end{verbatim}
      }
    \end{eblock}
  \end{columns}
  \begin{columns}
    \column{0.5\textwidth}
    \begin{eblock}{}
      {\fontsize{4.5}{5.5}\selectfont
        \begin{verbatim}
~/Tutorials/BASH/scripts> cat helloname.sh
#!/bin/bash

# My Second Script

echo Please Enter your name:
read name1 name2
Greet="Welcome to HPC Training"
echo "Hello $name1 $name2, $Greet"
        \end{verbatim}
      }
    \end{eblock}
    \column{0.5\textwidth}
    \begin{eblock}{}
      {\fontsize{4.5}{5.5}\selectfont
        \begin{verbatim}
apacheco@apacheco:~/Tutorials/BASH/scripts> ./helloname.sh 
Please Enter your name:
Alex Pacheco
Hello Alex Pacheco, Welcome to HPC Training
        \end{verbatim}
      }
    \end{eblock}
  \end{columns}
\end{frame}

\section{Beyond Basic Shell Scripting}
\subsection{Arithmetic Operations}
\begin{frame}[fragile,allowframebreaks]
  \frametitle{\small Arithmetic Operations}
  \begin{itemize}
    \item You can carry out numeric operations on integer variables
%  \end{itemize}
  \begin{columns}
    \column{0.6\textwidth}
    \vspace{-0.8cm}
    \begin{center}
      \begin{tikzpicture}
        \node (tbl) {
          \begin{tabularx}{\textwidth}{ccl}
            \arrayrulecolor{tigersgold}
            \textcolor{white}{\textbf{Operation} }& \textcolor{white}{\textbf{Operator}} & \\
            Addition\rule{0pt}{3.5ex} & + & \\
            Subtraction & - & \\
            Multiplication & * & \\
            Division & / \\
            Exponentiation & ** & (\textbf{bash} only)\\
            Modulo & \% & \\ 
            [1.0ex]
        \end{tabularx}};
        \begin{pgfonlayer}{background}
          \draw[rounded corners,top color=blue!30!black,bottom color=blue!10!white,
            draw=tigerspurple!30] ($(tbl.north west)+(0.14,0)$)
          rectangle ($(tbl.north east)-(0.13,0.9)$);
          \draw[rounded corners,top color=green!5,bottom color=green!5,draw=green!5]
          ($(tbl.north east)-(0.13,0.6)$)
          rectangle ($(tbl.south west)+(0.13,0.2)$);
        \end{pgfonlayer}
      \end{tikzpicture}
    \end{center}
  \end{columns}
%  \begin{itemize}
    \item Arithmetic operations in \textbf{bash} can be done within the \texttt{\$(($\cdots$))} or \texttt{\$[$\cdots$]} commands
    \begin{enumerate}
      {\scriptsize
      \item[$\bigstar$] Add two numbers: \texttt{\$((1+2))}
      \item[$\bigstar$] Multiply two numbers: \texttt{\$[\$a*\$b]}
      \item[$\bigstar$] You can also use the \texttt{let} command: \texttt{let c=\$a-\$b}
      \item[$\bigstar$] or use the \texttt{expr} command: \texttt{c=`expr \$a - \$b`}
      }
    \end{enumerate}
    \framebreak
    \item In \textbf{tcsh},
    \begin{enumerate}
      {\scriptsize
      \item[$\bigstar$] Add two numbers: \texttt{@ x = 1 + 2}
      \item[$\bigstar$] Divide two numbers: \texttt{@ x = \$a / \$b}
      \item[$\bigstar$] You can also use the \texttt{expr} command: \texttt{set c = `expr \$a \% \$b`}
      }
    \end{enumerate}
    \item Note the use of space
    \item[\textbf{bash}] space required around operator in the \texttt{expr} command
    \item[\textbf{tcsh}] space required between @ and variable, around = and numeric operators. 
    \item You can also use C-style increment operators
    \item[\textbf{bash}] \texttt{let c+=1} or \texttt{let c-{}-}
    \item[\textbf{tcsh}] \texttt{@ x -= 1} or \texttt{@ x++}
    \item[] \texttt{/=}, \texttt{*=} and \texttt{\%=} are also allowed.
    \item[\textbf{bash}]
    \item The above examples only work for integers.
    \item What about floating point number?
    \framebreak
    \item Using floating point in \textbf{bash} or \textbf{tcsh} scripts requires an external calculator like GNU \texttt{bc}.
    \begin{enumerate}
      {\scriptsize
        \item[$\bigstar$] Add two numbers:
        \item[]\texttt{echo "3.8 + 4.2" | bc}
        \item[$\bigstar$] Divide two numbers and print result with a precision of 5 digits:
        \item[]\texttt{echo "scale=5; 2/5" | bc}
        \item[$\bigstar$] Call \texttt{bc} directly:
        \item[]\texttt{bc $<<<$ "scale=5; 2/5"}
        \item[$\bigstar$] Use \texttt{bc -l} to see result in floating point at max scale:
        \item[] \texttt{bc -l $<<<$ "2/5"}
      }
    \end{enumerate}
  \end{itemize}
  \begin{bblock}{Exercise}
    Write a script to add/subtract/multiply/divide two numbers.
  \end{bblock}
  \framebreak
  {\tiny
    \begin{columns}
      \column{0.45\textwidth}
      \begin{eblock}{}
      \begin{verbatim}
~/Tutorials/BASH/scripts> cat dosum.sh
#!/bin/bash

FIVE=5
SEVEN=7
echo "5 + 7 = " $FIVE + $SEVEN
echo "5 + 7 = " $(($FIVE + $SEVEN))
let SUM=$FIVE+$SEVEN
echo "sum of 5 & 7 is " $SUM
exit
      \end{verbatim}
      \end{eblock}
      \column{0.55\textwidth}
      \vspace{-0.2cm}
      \begin{block}{}
      \begin{verbatim}
~/Tutorials/BASH/scripts> cat doratio.csh
#!/bin/tcsh

set FIVE=5
set SEVEN=7
echo "5 / 7 = " $FIVE / $SEVEN
@ RATIO = $FIVE / $SEVEN
echo "ratio of 5 & 7 is " $RATIO
set ratio=`echo "scale=5 ; $FIVE/$SEVEN" | bc`
echo "ratio of 5 & 7 is " $ratio
exit
      \end{verbatim}
      \end{block}
    \end{columns}
    \begin{columns}
      \column{0.45\textwidth}
      \vspace{0.2cm}
      \begin{eblock}{}
      \begin{verbatim}
~/Tutorials/BASH/scripts> ./dosum.sh 
5 + 7 = 5 + 7
5 + 7 =  12
sum of 5 & 7 is  12
      \end{verbatim}
      \end{eblock}
      \column{0.55\textwidth}
      \begin{block}{}
      \begin{verbatim}
~/Tutorials/BASH/scripts> ./doratio.csh 
5 / 7 =  5 / 7
ratio of 5 & 7 is  0
ratio of 5 & 7 is  .71428
      \end{verbatim}
      \end{block}
    \end{columns}
  }
\end{frame}


%\begin{frame}<0>[fragile]
%  \frametitle{\small Example for doing backups}
%  {\tiny
%  \begin{verbatim}
%~/Tutorials/BASH/scripts> cat backups.sh 
%#!/bin/bash
%
%BACKUPDIR=$(pwd)
%OF=$BACKUPDIR/$(date +%Y-%m-%d).tgz
%tar -czf ${OF} ./*sh
%~/Tutorials/BASH/scripts> ./backups.sh 
%~/Tutorials/BASH/scripts> ls *gz
%2012-09-18.tgz
%  \end{verbatim}
%  }
%\end{frame}

\subsection{Arrays}
\begin{frame}[fragile,allowframebreaks]
  \frametitle{\small Arrays}
  \begin{itemize}
    \item \textbf{bash} and \textbf{tcsh} supports one-dimensional arrays.
    \item Array elements may be initialized with the \texttt{variable[xx]} notation
    \item[] \texttt{variable[xx]=1}
    \item Initialize an array during declaration
    \item[\textbf{bash}] \texttt{name=(firstname 'last name')}
    \item[\textbf{tcsh}] \texttt{set name = (firstname 'last name')}
    \item reference an element \texttt{i} of an array \texttt{name}
    \item[] \texttt{\$\{name[i]\}}
    \item print the whole array
    \item[\textbf{bash}] \texttt{\$\{name[@]\}}
    \item[\textbf{tcsh}] \texttt{\$\{name\}}
    \item print length of array
    \item[\textbf{bash}] \texttt{\$\{\#name[@]\}}
    \item[\textbf{tcsh}] \texttt{\$\{\#name\}}
    \item print length of element \texttt{i} of array \texttt{name}
    \item[] \texttt{\$\{\#name[i]\}}
    \item[] Note: In \textbf{bash} \texttt{\$\{\#name\}} prints the length of the first element of the array
    \item Add an element to an existing array
    \item[\textbf{bash}] \texttt{name=(title \$\{name[@]\})}
    \item[\textbf{tcsh}] \texttt{set name = ( title  "\$\{name\}")}
    \item In \textbf{tcsh} everything within "..." is one variable.
    \item In the above \textbf{tcsh} example, \texttt{title} is first element of new array while the second element is the old array \texttt{name}
    \item copy an array \texttt{name} to an array \texttt{user}
    \item[\textbf{bash}] \texttt{user=(\$\{name[@]\})}
    \item[\textbf{tcsh}] \texttt{set user = ( \$\{name\} )}
    \framebreak
    \item concatenate two arrays
    \item[\textbf{bash}] \texttt{nameuser=(\$\{name[@]\} \$\{user[@]\})}
    \item[\textbf{tcsh}] \texttt{set nameuser=( \$\{name\} \$\{user\} )}
    \item delete an entire array
    \item[] \texttt{unset name}
    \item remove an element \texttt{i} from an array
    \item[\textbf{bash}] \texttt{unset name[i]}
    \item[\textbf{tcsh}] \texttt{ @ j = \$i - 1}
    \item[ ] \texttt{ @ k =\$i + 1 }
    \item[ ] \texttt{ set name = ( \$\{name[1-\$j]\} \$\{name[\$k-]\})}
    \item[\textbf{bash}] the first array index is zero (0)
    \item[\textbf{tcsh}] the first array index is one (1)
  \end{itemize}
  \framebreak
  \begin{bblock}{Exercise}
    \begin{enumerate}
      \item Write a script to read your first and last name to an array.
      \item Add your salutation and suffix to the array.
      \item Drop either the salutation or suffix.
      \item Print the array after each of the three steps above.
    \end{enumerate}
  \end{bblock}
  \framebreak
  \fontsize{5}{5.5}\selectfont{
    \begin{columns}
      \column{0.5\textwidth}
      \vspace{-0.5cm}
      \begin{eblock}{}
        \begin{verbatim}
~/Tutorials/BASH/scripts> cat name.sh
#!/bin/bash

echo "Print your first and last name"
read firstname lastname

name=($firstname $lastname)

echo "Hello " ${name[@]}

echo "Enter your salutation"
read title

echo "Enter your suffix"
read suffix

name=($title "${name[@]}" $suffix)
echo "Hello " ${name[@]}

unset name[2]
echo "Hello " ${name[@]}
        \end{verbatim}
      \end{eblock}
    \column{0.5\textwidth}
      \vspace{-0.5cm}
      \begin{bblock}{}
        \begin{verbatim}
~/Tutorials/BASH/scripts> cat name.csh
#!/bin/tcsh

echo "Print your first name"
set firstname = $<
echo "Print your last name"
set lastname = $<

set name = ( $firstname $lastname)
echo "Hello " ${name}

echo "Enter your salutation"
set title = $<

echo "Enter your suffix"
set suffix = "$<"

set name = ($title $name $suffix )
echo "Hello " ${name}

@ i = $#name
set name = ( $name[1-2] $name[4-$i] )
echo "Hello " ${name}
        \end{verbatim}
      \end{bblock}
    \end{columns}
    \begin{columns}
      \column{0.5\textwidth}
      \begin{eblock}{}
        \begin{verbatim}
~/Tutorials/BASH/scripts> ./name.sh 
Print your first and last name
Alex Pacheco
Hello  Alex Pacheco
Enter your salutation
Dr.
Enter your suffix
the first
Hello  Dr. Alex Pacheco the first
Hello  Dr. Alex the first
        \end{verbatim}
      \end{eblock}
      \column{0.5\textwidth}
      \begin{bblock}{}
        \begin{verbatim}
~/Tutorials/BASH/scripts> ./name.csh 
Print your first name
Alex
Print your last name
Pacheco
Hello  Alex Pacheco
Enter your salutation
Dr.
Enter your suffix
the first
Hello  Dr. Alex Pacheco the first
Hello  Dr. Alex the first
        \end{verbatim}
      \end{bblock}
    \end{columns}
  }
\end{frame}

\subsection{Flow Control}
\begin{frame}
  \frametitle{\small Flow Control}
  \begin{itemize}
    \item Shell Scripting Languages execute commands in sequence similar to programming languages such as C, Fortran, etc.
    \item Control constructs can change the sequential order of commands.
    \item Control constructs available in \textbf{bash} and \textbf{tcsh} are
    \begin{enumerate}
%      {\scriptsize
        \item Conditionals: \texttt{if}
        \item Loops: \texttt{for, while, until}
        \item Switches: \texttt{case, switch}
%      }
    \end{enumerate}
  \end{itemize}
\end{frame}

\begin{frame}
  \frametitle{\small \texttt{if} statement}
  \begin{itemize}
    \item An \texttt{if/then} construct tests whether the exit status of a list of commands is 0, and if so, executes one or more commands.
    \begin{columns}
      \column{5cm}
      \begin{eblock}{bash}
        \begin{tabbing}
          \texttt{if} \=\textit{[ condition1 ]; } \texttt{then}\\
          \> \textit{some commands}\\
          \texttt{elif} \textit{ [ condition2 ]; } \texttt{then}\\
          \> \textit{some commands}\\
          \texttt{else}\\
          \> \textit{some commands}\\
          \texttt{fi}
        \end{tabbing}
      \end{eblock}
      \column{5cm}
      \begin{bblock}{tcsh}
        \begin{tabbing}
          \texttt{if} \=\textit{( condition1 ) } \texttt{then}\\
          \> \textit{some commands}\\
          \texttt{else if} \textit{ ( condition2 ) } \texttt{then}\\
          \> \textit{some commands}\\
          \texttt{else}\\
          \> \textit{some commands}\\
          \texttt{endif}
        \end{tabbing}
      \end{bblock}
    \end{columns}
  \item Note the space between \textit{condition} and "["\quad"]"
  \item \textbf{bash} is very strict about spaces.
  \item \textbf{tcsh} commands are not so strict about spaces.
  \item \textbf{tcsh} uses the \texttt{if-then-else if-else-endif} similar to Fortran.   
  \end{itemize}
\end{frame}

\begin{frame}
  \frametitle{\small File Test \& Logical Operators}
  \begin{columns}
    \column{1.1\textwidth}
    \begin{eblock}{File Test Operators}
      \begin{tabular}{lll}
        Operation & \textbf{bash} & \textbf{tcsh} \\
        \hline
        file exists & \texttt{if [ -e .bashrc ]} & \texttt{if ( -e .tcshrc )}\\
        file is a regular file & \texttt{if [ -f .bashrc ]} &  \\
        file is a directory & \texttt{if [ -d /home ]} & \texttt{if ( -d /home )} \\
        file is not zero size & \texttt{if [ -s .bashrc ]} & \texttt{if ( ! -z .tcshrc)} \\
        file has read permission & \texttt{ if [ -r .bashrc ]} & \texttt{ if ( -r .tcshrc)} \\
        file has write permission & \texttt{ if [ -w .bashrc ]} & \texttt{ if ( -w .tcshrc)} \\
        file has execute permission & \texttt{ if [ -x .bashrc ]} & \texttt{ if ( -x .tcshrc)} \\
        \hline
      \end{tabular}
    \end{eblock}
    \begin{eblock}{Logical Operators}
      \vspace{-0.2cm}
      \begin{columns}
        \column{2cm}
        \begin{itemize}
          \item[!]: NOT
          \item[\&\&]: AND
          \item[||]: OR
        \end{itemize}
        \column{9cm}
        \begin{itemize}
          \item[] \texttt{if [ ! -e .bashrc ]}
          \item[] \texttt{if [ -f .bashrc ] \&\& [ -s .bashrc ]}
          \item[] \texttt{if [[ -f .bashrc || -f .bash\_profile ]] }
          \item[] \texttt{if ( -e ~/.tcshrc  \&\&  ! -z ~/.tcshrc )}
        \end{itemize}
      \end{columns}
    \end{eblock}
  \end{columns}
\end{frame}

\begin{frame}
  \frametitle{\small Integer \& String Comparison Operators}
  \begin{columns}
    \column{\textwidth}
    \vspace{-0.5cm}
    \begin{eblock}{Integer Comparison}
      \begin{tabular}{lll}
        Operation & \textbf{bash} & \textbf{tcsh} \\
        \hline
        equal to & \texttt{if [ 1 -eq 2 ]} & \texttt{if (1 == 2)} \\
        not equal to & \texttt{if [ \$a -ne \$b ]} & \texttt{if (\$a != \$b)}\\
        greater than & \texttt{if [ \$a -gt \$b ]} & \texttt{if (\$a > \$b)}\\
        greater than or equal to & \texttt{if [ 1 -ge \$b ]} & \texttt{if (1 >= \$b)}\\
        less than & \texttt{if [ \$a -lt 2 ]} & \texttt{if (\$a < 2)}\\
        less than or equal to & \texttt{if [[ \$a -le \$b ]]} & \texttt{if (\$a <= \$b)} \\
        \hline
      \end{tabular}
    \end{eblock}
    \begin{eblock}{String Comparison}
      \begin{tabular}{lll}
        Operation & \textbf{bash} & \textbf{tcsh} \\
        \hline
        equal to & \texttt{if [ \$a == \$b ]} & \texttt{if (\$a == \$b)}\\
        not equal to & \texttt{if [ \$a != \$b ]} & \texttt{if (\$a != \$b)}\\
        zero length or null & \texttt{if [ -z \$a ] } & \texttt{if (\$\%a == 0)}\\
        non zero length & \texttt{if [ -n \$a ] } & \texttt{if (\$\%a > 0)}\\
        \hline
      \end{tabular}
    \end{eblock}
  \end{columns}
\end{frame}

%\begin{frame}<0>[fragile]
%  \frametitle{\small Examples \& More}
%  {\tiny
%  \begin{verbatim}
%~/Tutorials/BASH/scripts> cat backups2.sh
%#!/bin/bash
%
%OF=$(date +%Y-%m-%d).tgz
%
%if [ -e "$OF" ]; then
%  echo "You have already created a backup today"
%  echo `ls -ltr $OF`
%else
%  tar -czf ${OF} ./*sh
%fi
%~/Tutorials/BASH/scripts> ls
%2012-09-18.tgz  backups.csh  dosum.sh      hello.sh          name.sh    shift.sh
%backups2.sh     backups.sh   helloname.sh  hellovariable.sh  quotes.sh  tmp
%~/Tutorials/BASH/scripts> ./backups2.sh 
%~/Tutorials/BASH/scripts> ./backups2.sh 
%You have already created a backup today
%-rw-r--r-- 1 apacheco users 1168 Sep 24 13:16 2012-09-24.tgz
%~/Tutorials/BASH/scripts>
%  \end{verbatim}
%  }
%\end{frame}

\begin{frame}[fragile]{\small Examples}
  \begin{itemize}
    \item Condition tests using the \texttt{if/then} may be nested
    {\fontsize{4.5}{5.5}\selectfont
      \begin{columns}
        \column{0.55\textwidth}
        \begin{eblock}{}
      \begin{verbatim}
read a
if [ "$a" -gt 0 ]; then
  if [ "$a" -lt 5 ]; then
    echo "The value of \"a\" lies somewhere between 0 and 5"
  fi
fi
      \end{verbatim}
        \end{eblock}
        \column{0.5\textwidth}
        \begin{bblock}{}
      \begin{verbatim}
set a = $<
if ( $a > 0 ) then
  if ( $a < 5 ) then
    echo "The value of $a lies somewhere between 0 and 5"
  endif
endif
      \end{verbatim}
        \end{bblock}
      \end{columns}
    }
    \item This is same as 
    {\fontsize{4.5}{5.5}\selectfont
      \begin{columns}
        \column{0.55\textwidth}
        \begin{eblock}{}
      \begin{verbatim}
read a
if [[ "$a" -gt 0 &&  "$a" -lt 5 ]]; then
  echo "The value of $a lies somewhere between 0 and 5"
fi
OR
if [ "$a" -gt 0 ] && [ "$a" -lt 5 ]; then
  echo "The value of $a lies somewhere between 0 and 5"
fi
      \end{verbatim}
        \end{eblock}
        \column{0.5\textwidth}
        \begin{bblock}{}
      \begin{verbatim}
set a = $<
if ( "$a" > 0  &&  "$a" < 5 ) then
  echo "The value of $a lies somewhere between 0 and 5"
endif
      \end{verbatim}
        \end{bblock}
      \end{columns}
    }
  \end{itemize}
\end{frame}

\begin{frame}[fragile,allowframebreaks]
  \frametitle{\small Loop Constructs}
  \begin{itemize}
%    \fontsize{7}{9}\selectfont{
    \item A \textit{loop} is a block of code that iterates a list of commands as long as the \textit{loop control condition} is true.
    \item Loop constructs available in 
    \item[\textbf{bash}:] \texttt{for, while} and \texttt{until}
    \item[\textbf{tcsh}:] \texttt{foreach} and \texttt{while}
%    }
  \end{itemize}
  \framebreak
  \begin{eblock}{bash}
%    \fontsize{5}{6}\selectfont{
    \begin{itemize}
      \item The \texttt{for} loop is the basic looping construct in \textbf{bash}
        \begin{tabbing}
          \texttt{for} \=\textit{arg} in \textit{list}\\
          \texttt{do}\\
          \>\textit{some commands}\\
          \texttt{done}
        \end{tabbing}
      \item the \texttt{for} and \texttt{do} lines can be written on the same line: \texttt{for} \textit{arg} in \textit{list}; \texttt{do}
      \item \texttt{for} loops can also use C style syntax
      \begin{tabbing}
        \texttt{for} \=((\textit{EXP1; EXP2; EXP3} )); do\\
        \>\textit{some commands}\\
        \texttt{done}
      \end{tabbing}
    \end{itemize}
%    }
  \end{eblock}
  \begin{columns}
    \column{0.3\textwidth}
    \begin{eblock}{}
      \fontsize{7}{6}\selectfont{
        \begin{verbatim}
for i in $(seq 1 10)
do
  touch file${i}.dat
done
        \end{verbatim}
    }
    \end{eblock}
    \column{0.35\textwidth}
    \begin{eblock}{}
      \fontsize{7}{6}\selectfont{
        \begin{verbatim}
for i in $(seq 1 10); do
  touch file${i}.dat
done
        \end{verbatim}
    }
    \end{eblock}
    \column{0.3\textwidth}
    \begin{eblock}{}
      \fontsize{7}{6}\selectfont{
        \begin{verbatim}
for ((i=1;i<=10;i++))
do
  touch file${i}.dat
done
        \end{verbatim}
    }
    \end{eblock}
  \end{columns}

  \begin{bblock}{tcsh}
%    \fontsize{7}{6}\selectfont{
    \begin{itemize}
      \item The \texttt{foreach} loop is the basic looping construct in \textbf{tcsh}
        \begin{tabbing}
          \texttt{foreach} \=\textit{arg} \textit{(list)} \\
          \>\textit{some commands}\\
          \texttt{end}
        \end{tabbing}
    \end{itemize}
%    }
  \end{bblock}
  \begin{columns}
    \column{0.4\textwidth}
    \begin{bblock}{}
      \fontsize{7}{6}\selectfont{
        \begin{verbatim}
foreach i (`seq 1 10`)
  touch file$i.dat
end
        \end{verbatim}
      }
    \end{bblock}
  \end{columns}
  \begin{eblock}{\texttt{while} loop}
    \begin{itemize}
%      \fontsize{7}{9}\selectfont{
      \item The \texttt{while} construct tests for a condition at the top of a loop, and keeps looping as long as that condition is true (returns a 0 exit status). 
      \item In contrast to a \texttt{for} loop, a \texttt{while} loop finds use in situations where the number of loop repetitions is not known beforehand.
      \begin{columns}
        \column{0.35\textwidth}
      \item \texttt{bash}
        \begin{tabbing}
          \texttt{while} \=[ \textit{condition} ]\\
          \texttt{do}\\
          \>\textit{some commands}\\
          \texttt{done}
        \end{tabbing}
        \column{0.35\textwidth}
      \item \texttt{tcsh}
        \begin{tabbing}
          \texttt{while} \=( \textit{condition} ) \\
          \>\textit{some commands}\\
          \texttt{end}
        \end{tabbing}
      \end{columns}
%      }
     \end{itemize}
  \end{eblock}
  \begin{columns}
    \column{0.4\textwidth}
    \vspace{-0.2cm}
    \begin{eblock}{}
      \fontsize{4.5}{5.5}\selectfont{
        \begin{verbatim}
#!/bin/bash

read counter
factorial=1
while [ $counter -gt 0 ]
do
   factorial=$(( $factorial * $counter ))
   counter=$(( $counter - 1 ))
done
echo $factorial
        \end{verbatim}
      }
    \end{eblock}
    \column{0.4\textwidth}
    \vspace{-0.2cm}
    \begin{bblock}{}
      \fontsize{4.5}{5.5}\selectfont{
        \begin{verbatim}
#!/bin/tcsh

set counter=$<
set factorial=1
while ( $counter > 0 )
   @ factorial = $factorial * $counter
   @ counter -= 1
end
echo $factorial
        \end{verbatim}
      }
    \end{bblock}
  \end{columns}
  \begin{eblock}{\texttt{until} loop}
    \begin{itemize}
%      \fontsize{7}{9}\selectfont{
      \item The \texttt{until} construct tests for a condition at the top of a loop, and keeps looping as long as that condition is false (opposite of \texttt{while} loop).
        \begin{tabbing}
          \texttt{until} \=[ \textit{condition is true} ]\\
          \texttt{do}\\
          \>\textit{some commands}\\
          \texttt{done}
        \end{tabbing}
%      }
    \end{itemize}
  \end{eblock}
  \vspace{-0.2cm}
  \begin{columns}
    \column{0.4\textwidth}
    \begin{eblock}{}
      \fontsize{5}{6}\selectfont{
        \begin{verbatim}
#!/bin/bash

read counter
factorial=1
until [ $counter -le 1 ]; do                                                                                                                                         
  factorial=$[ $factorial * $counter ]
  if [ $counter -eq 2 ]; then 
    break
  else
    let counter-=2
  fi                                                                                                                                        
done                                                                                                                                                            
echo $factorial
        \end{verbatim}
      }
    \end{eblock}
  \end{columns}
  \framebreak
  \begin{itemize}
    \item \texttt{for, while \& until} loops can nested. To exit from the loop use the \texttt{break} command
  \end{itemize}
  \fontsize{5}{5}\selectfont{
    \begin{columns}
      \column{0.45\textwidth}
      \vspace{-0.2cm}
      \begin{eblock}{}
        \begin{verbatim}
~/Tutorials/BASH/scripts> cat nestedloops.sh
#!/bin/bash

## Example of Nested loops

echo "Nested for loops"
for a in $(seq 1 5) ; do
  echo "Value of a in outer loop:" $a
  for b in `seq 1 2 5` ; do
    c=$(($a*$b))
    if [ $c -lt 10 ]; then
      echo "a * b = $a * $b = $c"
    else
      echo "$a * $b > 10"
      break
    fi
  done
done
echo "========================"
echo
echo "Nested for and while loops"
for ((a=1;a<=5;a++)); do
  echo "Value of a in outer loop:" $a
  b=1
  while [ $b -le 5 ]; do
    c=$(($a*$b))
    if [ $c -lt 5 ]; then
      echo "a * b = $a * $b = $c"
    else
      echo "$a * $b > 5"
      break
    fi
    let b+=2
  done
done
echo "========================"
        \end{verbatim}
      \end{eblock}
      \column{0.45\textwidth}
      \vspace{-0.2cm}
      \begin{bblock}{}
        \begin{verbatim}
~/Tutorials/BASH/scripts> cat nestedloops.csh
#!/bin/tcsh

## Example of Nested loops

echo "Nested for loops"
foreach a (`seq 1 5`)
  echo "Value of a in outer loop:" $a
  foreach b (`seq 1 2 5`)
    @ c = $a * $b
    if ( $c <  10 ) then
      echo "a * b = $a * $b = $c"
    else
      echo "$a * $b > 10"
      break
    endif
  end
end
echo "========================"
echo
echo "Nested for and while loops"
foreach a (`seq 1 5`)
  echo "Value of a in outer loop:" $a
  set b = 1
  while ( $b <= 5 )
    @ c = $a * $b
    if ( $c < 5 ) then
      echo "a * b = $a * $b = $c"
    else
      echo "$a * $b > 5"
      break
    endif
    @ b = $b + 2
  end
end
echo "========================"
          \end{verbatim}
      \end{bblock}
    \end{columns}
    \framebreak
    \begin{columns}
      \column{0.45\textwidth}
      \vspace{-0.2cm}
      \begin{eblock}{}
        \begin{verbatim}
~/Tutorials/BASH/scripts> ./nestedloops.sh 
Nested for loops
Value of a in outer loop: 1
a * b = 1 * 1 = 1
a * b = 1 * 3 = 3
a * b = 1 * 5 = 5
Value of a in outer loop: 2
a * b = 2 * 1 = 2
a * b = 2 * 3 = 6
2 * 5 > 10
Value of a in outer loop: 3
a * b = 3 * 1 = 3
a * b = 3 * 3 = 9
3 * 5 > 10
Value of a in outer loop: 4
a * b = 4 * 1 = 4
4 * 3 > 10
Value of a in outer loop: 5
a * b = 5 * 1 = 5
5 * 3 > 10
========================

Nested for and while loops
Value of a in outer loop: 1
a * b = 1 * 1 = 1
a * b = 1 * 3 = 3
1 * 5 > 5
Value of a in outer loop: 2
a * b = 2 * 1 = 2
2 * 3 > 5
Value of a in outer loop: 3
a * b = 3 * 1 = 3
3 * 3 > 5
Value of a in outer loop: 4
a * b = 4 * 1 = 4
4 * 3 > 5
Value of a in outer loop: 5
5 * 1 > 5
========================
        \end{verbatim}
      \end{eblock}
      \column{0.45\textwidth}
      \vspace{-0.2cm}
      \begin{bblock}{}
        \begin{verbatim}
~/Tutorials/BASH/scripts> ./nestedloops.csh 
Nested for loops
Value of a in outer loop: 1
a * b = 1 * 1 = 1
a * b = 1 * 3 = 3
a * b = 1 * 5 = 5
Value of a in outer loop: 2
a * b = 2 * 1 = 2
a * b = 2 * 3 = 6
2 * 5 > 10
Value of a in outer loop: 3
a * b = 3 * 1 = 3
a * b = 3 * 3 = 9
3 * 5 > 10
Value of a in outer loop: 4
a * b = 4 * 1 = 4
4 * 3 > 10
Value of a in outer loop: 5
a * b = 5 * 1 = 5
5 * 3 > 10
========================

Nested for and while loops
Value of a in outer loop: 1
a * b = 1 * 1 = 1
a * b = 1 * 3 = 3
1 * 5 > 5
Value of a in outer loop: 2
a * b = 2 * 1 = 2
2 * 3 > 5
Value of a in outer loop: 3
a * b = 3 * 1 = 3
3 * 3 > 5
Value of a in outer loop: 4
a * b = 4 * 1 = 4
4 * 3 > 5
Value of a in outer loop: 5
5 * 1 > 5
========================
        \end{verbatim}
      \end{bblock}
    \end{columns}
  }
\end{frame}

\begin{frame}[fragile,allowframebreaks]
  \frametitle{\small Switching or Branching Constructs}
  \begin{itemize}
%    \fontsize{7}{9}\selectfont{
    \item The \texttt{case} and \texttt{select} constructs are technically not loops, since they do not iterate the execution of a code block. 
    \item Like loops, however, they direct program flow according to conditions at the top or bottom of the block.
    \begin{columns}
      \column{5cm}
      \begin{eblock}{\texttt{case} construct}
        \begin{tabbing}
          \texttt{case} \=\textit{"\$variable"} in\\
          \>\textit{"\$condition1")}\\
          \>\textit{some command}\\
          \>;;\\
          \>\textit{"\$condition2")}\\
          \>some other commands\\
          \>;;\\
          \texttt{esac}
        \end{tabbing}
      \end{eblock}
      \column{5cm}
      \begin{eblock}{\texttt{select} construct}
        \begin{tabbing}
          \texttt{select} \=\textit{variable} [in list]\\
          \texttt{do}\\
          \>\textit{command}\\
          \>\texttt{break}\\
          \texttt{done}
        \end{tabbing}
      \end{eblock}
    \end{columns}
    \framebreak
    \item \texttt{tcsh} has the \texttt{switch} construct
    \begin{columns}
      \column{5cm}
      \begin{eblock}{\texttt{switch} construct}
        \begin{tabbing}
          \texttt{switch} \=(\textit{arg list})\\
          \texttt{case} \textit{"\$variable"}\\
          \>\textit{some command}\\
          \>\texttt{breaksw}\\
          \texttt{end}
        \end{tabbing}
      \end{eblock}
    \end{columns}
%    }
  \end{itemize}
  \framebreak
  \begin{columns}
    \column{0.4\textwidth}
    \vspace{-0.5cm}
    \begin{eblock}{}
      {\fontsize{4}{5}\selectfont
        \begin{verbatim}
~/Tutorials/BASH/scripts> cat dooper.sh
#!/bin/bash

echo "Print two numbers"
read num1 num2
echo "What operation do you want to do?"
echo "Enter +, -, *, /, **, % or all"
read oper

case $oper in
  "+")
     echo "$num1 + $num2 =" $[$num1 + $num2]
     ;;
  "-")
     echo "$num1 - $num2 =" $[$num1 - $num2]
     ;;
  "*")
     echo "$num1 * $num2 =" $[$num1 * $num2]
     ;;
  "**")
     echo "$num1 ** $num2 =" $[$num1 ** $num2]
     ;;
  "/")
     echo "$num1 / $num2 =" $[$num1 / $num2]
     ;;
  "%")
     echo "$num1 % $num2 =" $[$num1 % $num2]
     ;;
  "all")
     echo "$num1 + $num2 =" $[$num1 + $num2]
     echo "$num1 - $num2 =" $[$num1 - $num2]
     echo "$num1 * $num2 =" $[$num1 * $num2]
     echo "$num1 ** $num2 =" $[$num1 ** $num2]
     echo "$num1 / $num2 =" $[$num1 / $num2]
     echo "$num1 % $num2 =" $[$num1 % $num2]
     ;;
   *)
     echo ``What do you want to do again?''
     ;;
esac
        \end{verbatim}
      }
    \end{eblock}
    \column{0.4\textwidth}
    \vspace{-0.5cm}
    \begin{bblock}{}
      {\fontsize{4}{4.5}\selectfont
        \begin{verbatim}
~/Tutorials/BASH/scripts> cat dooper.csh
#!/bin/tcsh


echo "Print two numbers one at a time"
set num1 = $<
set num2 = $<
echo "What operation do you want to do?"
echo "Enter +, -, x, /, % or all"
set oper = $<

switch ( $oper )
  case "x"
     @ prod = $num1 * $num2
     echo "$num1 * $num2 = $prod"
     breaksw
  case "all"
     @ sum = $num1 + $num2
     echo "$num1 + $num2 = $sum"
     @ diff = $num1 - $num2
     echo "$num1 - $num2 = $diff"
     @ prod = $num1 * $num2
     echo "$num1 * $num2 = $prod"
     @ ratio = $num1 / $num2
     echo "$num1 / $num2 = $ratio"
     @ remain = $num1 % $num2
     echo "$num1 % $num2 = $remain"
     breaksw
  case "*"
     @ result = $num1 $oper $num2
     echo "$num1 $oper $num2 = $result"
     breaksw
endsw
        \end{verbatim}
      }
    \end{bblock}
  \end{columns}
  \framebreak
  \begin{columns}
    \column{0.5\textwidth}
    \begin{eblock}{}
      {\fontsize{4}{5.5}\selectfont
        \begin{verbatim}
~/Tutorials/BASH/scripts> ./dooper.sh 
Print two numbers
1 4
What operation do you want to do?
Enter +, -, *, /, **, % or all
all
1 + 4 = 5
1 - 4 = -3
1 * 4 = 4
1 ** 4 = 1
1 / 4 = 0
1 % 4 = 1
        \end{verbatim}
      }
    \end{eblock}
    \column{0.4\textwidth}
    \begin{bblock}{}
      {\fontsize{4}{5.5}\selectfont
        \begin{verbatim}
~/Tutorials/BASH/scripts> ./dooper.csh 
Print two numbers one at a time
1
5
What operation do you want to do?
Enter +, -, x, /, % or all
all
1 + 5 = 6
1 - 5 = -4
1 * 5 = 5
1 / 5 = 0
1 % 5 = 1
        \end{verbatim}
      }
    \end{bblock}
  \end{columns}
\end{frame}


\subsection{Command Line Arguments}
\begin{frame}[fragile,allowframebreaks]
  \frametitle{\small Command Line Arguments}
  \vspace{-0.5cm}
  \begin{columns}
    \column{12cm}
  \begin{itemize}
    \item Similar to programming languages, \texttt{bash} (and other shell scripting languages) can also take command line arguments
    \begin{enumerate}
      \fontsize{7}{10}\selectfont{
      \item[$\vardiamond$] \texttt{./scriptname arg1 arg2 arg3 arg4 ...}
      \item[$\vardiamond$] \texttt{\$0,\$1,\$2,\$3, etc}: positional parameters corresponding to \texttt{./scriptname,arg1,arg2,arg3,arg4,...} respectively  
      \item[$\vardiamond$] \texttt{\$\#}: number of command line arguments
      \item[$\vardiamond$] \texttt{\$*}: all of the positional parameters, seen as a single word
      \item[$\vardiamond$] \texttt{\$@}: same as \texttt{\$*} but each parameter is a quoted string.
      \item[$\vardiamond$] \texttt{shift N}: shift positional parameters from \texttt{N+1} to \texttt{\$\#} are renamed to variable names from \texttt{\$1} to \texttt{\$\# - N + 1}
      }
    \end{enumerate}
    \item In \texttt{csh,tcsh}
    \begin{enumerate}
      \fontsize{7}{10}\selectfont{
      \item[$\bigstar$] an array \texttt{argv} contains the list of arguments with \texttt{argv[0]} set to name of script.
      \item[$\bigstar$] \texttt{\#argv} is the number of arguments i.e. length of \texttt{argv} array.
      }
    \end{enumerate}
  \end{itemize}
  \end{columns}
  \framebreak
  \begin{columns}
    \column{0.5\textwidth}
    \vspace{-0.5cm}
    \begin{eblock}{}
      \fontsize{4}{5.5}\selectfont{
        \begin{verbatim}
~/Tutorials/BASH/scripts> cat shift.sh
#!/bin/bash

USAGE="USAGE: $0 <at least 1 argument>

if [[ "$#" -lt 1 ]]; then
   echo $USAGE
   exit
fi

echo "Number of Arguments: " $#
echo "List of Arguments: " $@
echo "Name of script that you are running: " $0
echo "Command You Entered:" $0 $*

while [ "$#"  -gt 0 ]; do
  echo "Argument List is: " $@
  echo "Number of Arguments: " $#
  shift 
done
        \end{verbatim}
      }
    \end{eblock}
    \column{0.5\textwidth}
    \vspace{-0.5cm}
    \begin{bblock}{}
      \fontsize{4}{5.5}\selectfont{
        \begin{verbatim}
~/Tutorials/BASH/scripts> cat shift.csh
#!/bin/tcsh

set USAGE="USAGE: $0 <at least 1 argument>"

if ( "$#argv" < 1 ) then
   echo $USAGE
   exit
endif

echo "Number of Arguments: " $#argv
echo "List of Arguments: " $*
echo "Name of script that you are running: " $0
echo "Command You Entered:" $0 $*

while ( "$#argv"  > 0 )
  echo "Argument List is: " $*
  echo "Number of Arguments: " $#argv
  shift 
end
        \end{verbatim}
      }
    \end{bblock}
  \end{columns}
  \begin{columns}
    \column{0.5\textwidth}
    \begin{eblock}{}
      \fontsize{4}{5.5}\selectfont{
        \begin{verbatim}
~/Tutorials/BASH/scripts> ./shift.sh $(seq 1 5)
Number of Arguments:  5
List of Arguments:  1 2 3 4 5
Name of script that you are running:  ./shift.sh
Command You Entered: ./shift.sh 1 2 3 4 5
Argument List is:  1 2 3 4 5
Number of Arguments:  5
Argument List is:  2 3 4 5
Number of Arguments:  4
Argument List is:  3 4 5
Number of Arguments:  3
Argument List is:  4 5
Number of Arguments:  2
Argument List is:  5
Number of Arguments:  1
        \end{verbatim}
      }
    \end{eblock}
    \column{0.5\textwidth}
    \begin{bblock}{}
      \fontsize{4}{5.5}\selectfont{
        \begin{verbatim}
~/Tutorials/BASH/scripts> ./shift.csh $(seq 1 5)
Number of Arguments:  5
List of Arguments:  1 2 3 4 5
Name of script that you are running:  ./shift.csh
Command You Entered: ./shift.csh 1 2 3 4 5
Argument List is:  1 2 3 4 5
Number of Arguments:  5
Argument List is:  2 3 4 5
Number of Arguments:  4
Argument List is:  3 4 5
Number of Arguments:  3
Argument List is:  4 5
Number of Arguments:  2
Argument List is:  5
Number of Arguments:  1
        \end{verbatim}
      }
    \end{bblock}
  \end{columns}
\end{frame}

\begin{frame}[fragile,allowframebreaks]
  \frametitle{\small Scripting for Job Submission}
  \begin{bblock}{Problem Description}
    \begin{itemize}
      \item I have to run more than one serial job.
      \item I don't want to submit multiple job using the serial queue
      \item How do I submit \emph{one} job which can run multiple serial jobs?
    \end{itemize}
  \end{bblock}
  \begin{eblock}{Solution}
    \begin{itemize}
      \item Write a script which will log into all unique nodes and run your serial jobs in background.
      \item Easy said than done
      \item What do you need to know?
      \begin{enumerate}
        \fontsize{8}{7}\selectfont{
          \item Shell Scripting
          \item How to run a job in background
          \item Know what the \texttt{wait} command does
        }
      \end{enumerate}
    \end{itemize}
  \end{eblock}
  \framebreak
  \fontsize{5}{5.5}\selectfont{
    \begin{verbatim}
[apacheco@eric2 traininglab]$ cat checknodes.sh 
#!/bin/bash
#
#PBS -q checkpt
#PBS -l nodes=4:ppn=4
#PBS -l walltime=00:10:00
#PBS -V
#PBS -o nodetest.out
#PBS -e nodetest.err
#PBS -N testing
#

export WORK_DIR=$PBS_O_WORKDIR
export NPROCS=`wc -l $PBS_NODEFILE |gawk '//{print $1}'`
NODES=(`cat "$PBS_NODEFILE"` )
UNODES=(`uniq "$PBS_NODEFILE"` )

echo "Nodes Available: " ${NODES[@]}
echo "Unique Nodes Available: " ${UNODES[@]}

echo "Get Hostnames for all processes"
i=0
for nodes in "${NODES[@]}"; do
  ssh -n $nodes 'echo $HOSTNAME '$i' ' &
  let i=i+1
done
wait

echo "Get Hostnames for all unique nodes"
i=0
NPROCS=`uniq $PBS_NODEFILE | wc -l |gawk '//{print $1}'`
let NPROCS-=1
while [ $i -le $NPROCS ] ; do
  ssh -n ${UNODES[$i]} 'echo $HOSTNAME '$i' '
  let i=i+1
done
[apacheco@eric2 traininglab]$ qsub checknodes.sh 
[apacheco@eric2 traininglab]$ cat nodetest.out 
--------------------------------------
Running PBS prologue script
--------------------------------------
User and Job Data:
--------------------------------------
Job ID:    422409.eric2
Username:  apacheco
Group:     loniadmin
Date:      25-Sep-2012 11:01
Node:      eric010 (3053)
--------------------------------------
PBS has allocated the following nodes:

eric010
eric012
eric013
eric026

A total of 16 processors on 4 nodes allocated
---------------------------------------------
Check nodes and clean them of stray processes
---------------------------------------------
Checking node eric010 11:01:52 
Checking node eric012 11:01:54 
Checking node eric013 11:01:56 
Checking node eric026 11:01:57 
Done clearing all the allocated nodes
------------------------------------------------------
Concluding PBS prologue script - 25-Sep-2012 11:01:57
------------------------------------------------------
Nodes Available:  eric010 eric010 eric010 eric010 eric012 eric012 eric012 eric012 eric013 eric013 eric013 eric013 eric026 eric026 
eric026 eric026
Unique Nodes Available:  eric010 eric012 eric013 eric026
Get Hostnames for all processes
eric010 3
eric012 5
eric010 1
eric012 6
eric012 4
eric013 10
eric010 2
eric012 7
eric013 8
eric013 9
eric026 15
eric013 11
eric010 0
eric026 13
eric026 12
eric026 14
Get Hostnames for all unique nodes
eric010 0
eric012 1
eric013 2
eric026 3
------------------------------------------------------
Running PBS epilogue script    - 25-Sep-2012 11:02:00
------------------------------------------------------
Checking node eric010 (MS)
Checking node eric026 ok
Checking node eric013 ok
Checking node eric012 ok
Checking node eric010 ok
------------------------------------------------------
Concluding PBS epilogue script - 25-Sep-2012 11:02:06
------------------------------------------------------
Exit Status:    
Job ID:          422409.eric2
Username:        apacheco
Group:           loniadmin
Job Name:        testing
Session Id:      3052
Resource Limits: ncpus=1,nodes=4:ppn=4,walltime=00:10:00
Resources Used:  cput=00:00:00,mem=5260kb,vmem=129028kb,walltime=00:00:01
Queue Used:      checkpt
Account String:  loni_loniadmin1
Node:            eric010
Process id:      4101
------------------------------------------------------
    \end{verbatim}
  }
\end{frame}

\section{Advanced Topics}
\begin{frame}[fragile]{\small Declare command}
  \begin{itemize}
    \item Use the \textbf{declare} command to set variable and functions attributes.
    \item Create a constant variable i.e. read only variable
    \item[Syntax:] \texttt{declare -r var}
    \item[] \texttt{declare -r varName=value}
    \item Create an integer variable
    \item[Syntax:] \texttt{declare -i var}
    \item[] \texttt{declare -i varName=value}
    \item You can carry out arithmetic operations on variables declared as integers
  \end{itemize}
  \begin{columns}
    \column{5cm}
    {\fontsize{4.5}{5.5}\selectfont
      \begin{eblock}{}
        \begin{verbatim}
~/Tutorials/BASH> j=10/5 ; echo $j
10/5
~/Tutorials/BASH> declare -i j; j=10/5 ; echo $j
2
        \end{verbatim}
      \end{eblock}
    }
  \end{columns}
\end{frame}
\subsection{Functions}
\begin{frame}[fragile,allowframebreaks]{\small Functions}
  \begin{itemize}
    \item Like "real" programming languages, \textbf{bash} has functions.
    \item A function is a subroutine, a code block that implements a set of operations, a "black box" that performs a specified task. 
    \item Wherever there is repetitive code, when a task repeats with only slight variations in procedure, then consider using a function.
  \end{itemize}
  \begin{columns}
    \column{5cm}
    {\scriptsize
      \begin{eblock}{}
        \begin{tabbing}
          \textbf{fu}\=\textbf{nction} \texttt{function\_name} \{\\
          \>\textit{command}\\
          \}\\
          OR\\
          \texttt{function\_name} () \{\\
          \>\textit{command}\\
          \}
        \end{tabbing}
      \end{eblock}
    }
  \end{columns}
  \framebreak
  \begin{columns}
    \column{0.47\textwidth}
    {\fontsize{4.5}{5.5}\selectfont
      \begin{eblock}{}
        \begin{verbatim}
~/Tutorials/BASH/scripts> cat shift10.sh
#!/bin/bash

usage () {
        echo "USAGE: $0 [atleast 11 arguments]"
  exit
}

[[ "$#" -lt 11 ]] && usage

echo "Number of Arguments: " $#
echo "List of Arguments: " $@
echo "Name of script that you are running: " $0
echo "Command You Entered:" $0 $*
echo "First Argument" $1
echo "Tenth and Eleventh argument" $10 $11 ${10} ${11}

echo "Argument List is: " $@
echo "Number of Arguments: " $#
shift 9
echo "Argument List is: " $@
echo "Number of Arguments: " $#
        \end{verbatim}
      \end{eblock}
    }
    \column{0.53\textwidth}
    {\fontsize{4.5}{5.5}\selectfont
      \begin{eblock}{}
        \begin{verbatim}
~/Tutorials/BASH/scripts> ./shift10.sh `seq 1 2 22`
Number of Arguments:  11
List of Arguments:  1 3 5 7 9 11 13 15 17 19 21
Name of script that you are running:  ./shift10.sh
Command You Entered: ./shift10.sh 1 3 5 7 9 11 13 15 17 19 21
First Argument 1
Tenth and Eleventh argument 10 11 19 21
Argument List is:  1 3 5 7 9 11 13 15 17 19 21
Number of Arguments:  11
Argument List is:  19 21
Number of Arguments:  2
        \end{verbatim}
      \end{eblock}
    }
  \end{columns}
  \framebreak
  \begin{itemize}
    \item You can also pass arguments to a function.
    \item All function parameters or arguments can be accessed via \$1, \$2, \$3,..., \$N.
    \item \$0 always point to the shell script name.
    \item \$* or \$@ holds all parameters or arguments passed to the function.
    \item \$\# holds the number of positional parameters passed to the function.
    \item Array variable called \texttt{FUNCNAME} contains the names of all shell functions currently in the execution call stack.
    \item By default all variables are global.
    \item Modifying a variable in a function changes it in the whole script.
    \item You can create a local variables using the \textbf{local} command
    \item[Syntax:] \texttt{local var=value}
    \item[] \texttt{local varName}
  \end{itemize}
  \framebreak
%  \begin{columns}
%    \column{0.8\textwidth}
%    {\fontsize{4.5}{5.5}\selectfont
%      \begin{eblock}{Backup Script using Functions and Command line arguments}
%        \begin{verbatim}
%~/Tutorials/BASH/scripts> cat backups.sh
%#!/bin/bash
%
%backup () {
%        local d="$1"
%        [[ -z $d ]] && { echo "${FUNCNAME}(): directory name not specified"; exit 1; }
%        echo "Starting backup ..."
%        tar -cvzf $2 $d
%        echo "backup written to $2"
%}
%
%backup $1 $2
%        \end{verbatim}
%      \end{eblock}
%    }
%  \end{columns}
  \begin{itemize}
    \item A function may recursively call itself even without use of local variables.
  \end{itemize}
  \begin{columns}
    \column{0.5\textwidth}
    {\fontsize{4.5}{5.5}\selectfont
      \vspace{-0.35cm}
      \begin{eblock}{}
        \begin{verbatim}
~/Tutorials/BASH/scripts> cat factorial3.sh
#!/bin/bash

usage () {
  echo "USAGE: $0 <integer>"
  exit
}

factorial() {
  local i=$1
  local f

  declare -i i
  declare -i f

  if [[ "$i" -le 2 ]]; then
    echo $i
  else
    f=$(( $i - 1 ))
    f=$( factorial $f )
    f=$(( $f * $i ))
    echo $f
  fi
}

if [[ "$#" -eq 0 ]]; then
  usage
else
  j=1
  while [ $j -le $# ]; do
    x=$( factorial $j )
    echo "Factorial of $j is $x"
    let j++
  done
fi
        \end{verbatim}
      \end{eblock}
    }
    \column{0.5\textwidth}
      \vspace{-0.35cm}
    {\fontsize{4.5}{5.5}\selectfont
      \begin{eblock}{}
        \begin{verbatim}
~/Tutorials/BASH/scripts>./factorial3.sh $(seq 1 10)
Factorial of 1 is 1
Factorial of 2 is 2
Factorial of 3 is 6
Factorial of 4 is 24
Factorial of 5 is 120
Factorial of 6 is 720
Factorial of 7 is 5040
Factorial of 8 is 40320
Factorial of 9 is 362880
Factorial of 10 is 3628800
        \end{verbatim}
      \end{eblock}
    }
  \end{columns}
\end{frame}

\subsection{Regular Expressions}
\begin{frame}[allowframebreaks]
  \frametitle{\small Regular Expressions}
  \begin{itemize}
%    \fontsize{7}{9}\selectfont{
    \item A regular expression (regex) is a method of representing a string matching pattern. 
    \item Regular expressions enable strings that match a particular pattern within textual data records to be located and modified and they are often used within utility programs and programming languages that manipulate textual data. 
    \item Regular expressions are extremely powerful.
    \item Supporting Software and Tools
    \begin{enumerate}
      \fontsize{7}{10}\selectfont{
        \item Command Line Tools: grep, egrep, sed
        \item Editors: ed, vi, emacs
        \item Languages: awk, perl, python, php, ruby, tcl, java, javascript, .NET
      }
    \end{enumerate}
%    }
  \end{itemize}
  \begin{eblock}{Shell regex}
    \begin{itemize}
%      \fontsize{7}{9}\selectfont{
      \item[?]: match any single character.
      \item[*]: match zero or more characters.
      \item[{[ ]}]: match list of characters in the list specified
      \item[{[! ]}]: match characters not in the list specified
      \item[\^{}]: match at begining of line
      \item[\$]: match at end of line
      \item[{[\^{} ]}]: match characters not in the list specified
%      }
    \end{itemize}
  \end{eblock}
\end{frame}

\subsection{grep}
\begin{frame}
  \frametitle{\small grep \& egrep}
  \begin{itemize}
%    \fontsize{7}{9}\selectfont{
    \item \texttt{grep} is a Unix utility that searches through either information piped to it or files in the current directory.
    \item \texttt{egrep} is extended grep, same as \texttt{grep -E}
    \item Use \texttt{zgrep} for compressed files.
    \item Usage: \texttt{grep <options> <search pattern> <files>} 
    \item Commonly used options
    \begin{itemize}
      {\scriptsize
      \item[-i]: ignore case during search
      \item[-r]: search recursively
      \item[-v]: invert match i.e. match everything except pattern
      \item[-l]: list files that match pattern
      \item[-L]: list files that do not match pattern
      \item[-n]: prefix each line of output with the line number within its input file.
      }
    \end{itemize}
%    }
  \end{itemize}
\end{frame}

\subsection{awk primer}
\begin{frame}
  \frametitle{\small awk}
  \begin{itemize}
    \fontsize{7}{9}\selectfont{
    \item The Awk text-processing language is useful for such tasks as:
    \begin{enumerate}
      \fontsize{7}{9}\selectfont{
      \item[$\bigstar$] Tallying information from text files and creating reports from the results.
      \item[$\bigstar$]Adding additional functions to text editors like "vi".
      \item[$\bigstar$] Translating files from one format to another.
      \item[$\bigstar$] Creating small databases.
      \item[$\bigstar$]Performing mathematical operations on files of numeric data.
      }
    \end{enumerate}
    \item Awk has two faces: 
    \begin{enumerate}
      \fontsize{7}{9}\selectfont{
      \item[$\bigstar$] it is a utility for performing simple text-processing tasks, and 
      \item[$\bigstar$] it is a programming language for performing complex text-processing tasks.
      }
    \end{enumerate}
    \item Simplest form of using awk
    \begin{enumerate}
      \fontsize{7}{9}\selectfont{
      \item[$\vardiamond$]\textbf{awk} \textit{search pattern} \{\texttt{program actions}\}
      \item[$\vardiamond$] Most command action: \texttt{print}
      \item[$\vardiamond$] Print file dosum.sh: \texttt{awk '\{print \$0\}' dosum.sh}
      \item[$\vardiamond$] Print line matching bash in all files in current directory:
      \item[] \texttt{awk '/bash/\{print \$0\}' *.sh }
        }
    \end{enumerate}
    \item \textbf{awk} supports the if conditional and for loops
    \fontsize{7}{9}\selectfont{
    \item[] \texttt{awk '\{ if (NR > 0)\{print "File not empty"\}\}' hello.sh}
    \item[] \texttt{awk '\{for (i=1;i<=NF;i++)\{print \$i\}\}' name.sh}
    \item[] \texttt{ls *.sh | awk -F. '\{print \$1\}'}
    \item[] NR$\equiv$Number of records; NF$\equiv$Number of fields (or columns)
    }
    \item awk one-liners: \url{http://www.pement.org/awk/awk1line.txt}
    }
  \end{itemize}
\end{frame}

\subsection{sed primer}
\begin{frame}
  \frametitle{\small sed}
  \begin{itemize}
    \fontsize{7}{9}\selectfont{
    \item sed ("stream editor") is Unix utility for parsing and transforming text files.
    \item sed is line-oriented, it operates one line at a time and allows regular expression matching and substitution.
    \item The most commonly used feature of sed is the 's' (substitution command)
    \begin{enumerate}
      \fontsize{7}{9}\selectfont{
      \item[$\vardiamond$] echo Auburn Tigers | sed 's/Auburn/LSU/g'
      \item[$\bigstar$] Add the \texttt{-e} to carry out multiple matches.
      \item[$\vardiamond$] echo LSU Tigers | sed -e 's/LSU/LaTech/g' -e 's/Tigers/Bulldogs/g'
      \item[$\bigstar$] insert a blank line above and below the lines that match regex: 
      \item[]\texttt{sed '/regex/\{x;p;x;G;\}'}
      \item[$\bigstar$] delete all blank lines in a file: \texttt{sed '/\^{}\$/d'}
      \item[$\bigstar$] delete lines n through m in file: \texttt{sed 'n,md'}
      \item[$\bigstar$] delete lines matching pattern regex: \texttt{sed '/regex/d'}
      \item[$\bigstar$]print only lines which match regular expression: \texttt{sed -n '/regex/p'}
      \item[$\bigstar$] print section of file between two regex: \texttt{sed -n '/regex1/,/regex2/p'}
      \item[$\bigstar$] print section of file from regex to enf of file: \texttt{sed -n '/regex1/,\$p'}
      }
    \end{enumerate}
    \item sed one-liners: \url{http://sed.sourceforge.net/sed1line.txt}
    }
  \end{itemize}
\end{frame}

\section{Wrap Up}
\begin{frame}
  \frametitle{\small References \& Further Reading}
  \begin{itemize}
    \fontsize{7}{9}\selectfont{
    \item BASH Programming \url{http://tldp.org/HOWTO/Bash-Prog-Intro-HOWTO.html}
    \item Advanced Bash-Scripting Guide \url{http://tldp.org/LDP/abs/html/}
    \item Regular Expressions \url{http://www.grymoire.com/Unix/Regular.html}
    \item AWK Programming \url{http://www.grymoire.com/Unix/Awk.html}
    \item awk one-liners: \url{http://www.pement.org/awk/awk1line.txt}
    \item sed \url{http://www.grymoire.com/Unix/Sed.html}
    \item sed one-liners: \url{http://sed.sourceforge.net/sed1line.txt}
    \item CSH Programming \url{http://www.grymoire.com/Unix/Csh.html}
    \item csh Programming Considered Harmful \url{http://www.faqs.org/faqs/unix-faq/shell/csh-whynot/}
    \item Wiki Books \url{http://en.wikibooks.org/wiki/Subject:Computing}
    }
  \end{itemize}
\end{frame}

\begin{frame}
  \frametitle{\small Additional Help}
  \begin{itemize}
    \item User Guides
    \begin{enumerate}
      {\scriptsize
        \item[$\vardiamond$]LSU HPC: \url{http://www.hpc.lsu.edu/docs/guides.php\#hpc}
        \item[$\vardiamond$]LONI: \url{http://www.hpc.lsu.edu/docs/guides.php\#loni}
      }
    \end{enumerate}
    \item Documentation: \url{https://docs.loni.org}
    \item Online Courses: \url{https://docs.loni.org/moodle}
    \item Contact us
    \begin{enumerate}
      {\scriptsize
        \item[$\vardiamond$]Email ticket system: sys-help@loni.org
        \item[$\vardiamond$]Telephone Help Desk: 225-578-0900
        \item[$\vardiamond$]Instant Messenger (AIM, Yahoo Messenger, Google Talk)
        \begin{enumerate}
          {\scriptsize
            \item[$\bigstar$]Add "lsuhpchelp"
          }
        \end{enumerate}
      }
    \end{enumerate}
  \end{itemize}
\end{frame}

\begin{frame}
  \frametitle{}
  \color{DarkBlue}{
  \begin{center}
    {\fontsize{40}{60}\selectfont The End}\\
    \vspace{1cm}
    {\fontsize{20}{30}\selectfont Any Questions?}\\
    \vspace{0.5cm}
    {\fontsize{15}{30}\selectfont
      \color{red!80!black}{Next Week}\\      
      \color{red!80!black}{Introduction to Perl}\\
    }
    \vspace{0.5cm}
    {\fontsize{15}{20}\selectfont Survey: \color{red!90!white}{\url{http://www.hpc.lsu.edu/survey}}}
  \end{center}
  }
\end{frame}

\end{document}

